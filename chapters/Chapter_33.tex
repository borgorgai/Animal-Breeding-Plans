\chapter{Community Breeding}
\label{cha:Lush_Chapter_33}
\index{Community breeding|(}

Most breeds of livestock arose from community breeding in a small
region where a few herds located conveniently to each other exchanged
breeding stock during the formative period of the breed and really
established the breed by linebreeding to the best individuals within
those herds. Community breeding became less typical when herdbooks
were established and men unfamiliar with each other's stock could still
work with the same breed.

Community breeding has not been general in the United States,
although at one time the Poland-China breed was a community breed
in Butler and Warren Counties in Ohio. Likewise, the Chester-White
was long a community breed in Pennsylvania, although the details of
that history were not kept. A few other prominent breeds for at least
short periods have been community breeds in America before they
expanded to nationwide importance. Spread over a vast territory, with
breed organizations endeavoring to expand their spheres of influence,
and with high-pressure salesmanship often working most effectively on
prospects who are at some distance from the herds where the animals
are bred, the prominent breeds in America have generally been far from
any condition which could be called community breeding. Among scattered
examples which approached the condition of community breeding
in America should be mentioned the Vermont Merinos, light horses
in the Bluegrass region of Kentucky, the New Salem (North Dakota)
Breeding Circuit for Holstein-Friesians, and county associations such as
the Delaware County (Ohio) Percheron Breeders' Association and some
of the county associations of dairy breeders in Wisconsin.

Many writers on animal breeding subjects have emphasized the
advantages of community breeding, but this seems to have had little
effect on general practice. In many communities even yet a man beginning
to breed purebred livestock will deliberately select a breed which
is not present or at least is not abundant in his community, thinking
that he will thereby have less competition and a better chance to make
his herd well known than if he started with a breed already well
established in that community.

\section*{MORE ACCURATE CHOICE OF BREEDING STOCK}

One who sees his neighbor's herd frequently and knows many of the
animals in it, has a better chance to make correct choices in selecting
animals out of that herd than if he selects from a distant herd the first
time he sees it or if he buys in an auction sale an animal which may be
the only one there from the herd in which it was bred. In the neighboring
herd he has a chance to see at several different times or ages any
animal he is thinking of buying. Also, he can usually see many of its
close relatives. This is helpful in keeping to a minimum the amount he
is deceived by environment, by dominance, and by epistasis.

Moreover, he is dealing with a man whose business reputation he
knows and one who has a neighborly as well as a business interest in
keeping him satisfied with his bargain. If the purchase is unsatisfactory
and some adjustment is necessary to satisfy guarantees, there need be
little delay and no correspondence or traveling expense. The transportation
of the animals to be exchanged is a negligible item. Also, he can
more easily satisfy himself about the health of the herd from which he is
buying and thereby minimize the risk of introducing disease along with
his purchases.

\section*{PROMOTING THE CONTINUED USE OF GOOD SIRES}

One of the important advantages of community breeding is the ease
with which it leads to the exchange of sires thought to be unusually
good but which the present owner cannot use longer without close
inbreeding. If the general sentiment is in favor of using homebred
stock, then a sire proved unusually good by his offspring can be continued
in use in neighboring herds without extreme inbreeding. This
reaches an extreme form in the dairy bull circles discussed in
chapter~\ref{cha:Lush_Chapter_32}. Another fairly common form is the
stallion club for the joint ownership of a good stallion by several
farmers; although, if only one stallion is owned, this will not be much
help in keeping him in service more than three years. Cow-testing
associations are primarily organized to aid in the intelligent culling
of cows and in improving feeding practices but can lead to some community
breeding.

The exchange and continued use of the best proved sires in each
neighborhood, if carried far enough, will ultimately develop linebred
families which are distinct from one community to another. When that
has happened, comparisons of those families can be made, and weak
points of one can be corrected by mild outcrosses to others which are
strong in those points. The Homestead family of Holstein-Friesians was
a notable example of such community breeding.

Community breeding, if carried far, will tend to make the breeds
less uniform than they are today. Probably this would be a real advantage
to the utility of the breed, although interchange of breeding stock
from great distances would diminish and this would affect some of the
commercial aspects of the purebred business.

\section*{MUTUAL EDUCATION OF THE BREEDERS}
Where many people in one community are breeding the same kind
of livestock, there are frequent occasions for them to discuss their problems
of breeding, animal health, sales, shows, etc. If this continues long,
most of the people in that district soon know much about the lore of
that particular breed. Almost without realizing it, they come to possess
knowledge and skill ordinarily acquired by isolated breeders only after
years of experience. Something of this kind is seen in long-established
dairy regions and in the regions of Kentucky and Missouri where saddle
horses or Thoroughbreds are raised extensively. Where there is
much community interest in the breed, it is not difficult to arrange local
shows where the exhibits will be creditable and where lively interest
will exist, since nearly all of the animals shown will come from herds
which the spectators know personally.

\section*{MUTUAL EDUCATION OF THE BREEDERS}
Where many people in one community are breeding the same kind
of livestock, there are frequent occasions for them to discuss their problems
of breeding, animal health, sales, shows, etc. If this continues long,
most of the people in that district soon know much about the lore of
that particular breed. Almost without realizing it, they come to possess
knowledge and skill ordinarily acquired by isolated breeders only after
years of experience. Something of this kind is seen in long-established
dairy regions and in the regions of Kentucky and Missouri where saddle
horses or Thoroughbreds are raised extensively. Where there is
much community interest in the breed, it is not difficult to arrange local
shows where the exhibits will be creditable and where lively interest
will exist, since nearly all of the animals shown will come from herds
which the spectators know personally.

\section*{BUSINESS ASPECTS OF COMMUNITY BREEDING}

At present commercial necessities must govern the operations of
most breeders. Sometimes this operates against community breeding. A
prominent Jersey breeder says that many American-bred bulls are as
good as the average imported bull but at the same time advises young
breeders to use imported bulls to head their herds, since they will find a
readier sale for the young bull calves by an imported sire than if they
were by an American-bred bull. There is no sound genetic reason for
this. It is only that the word ``imported'' may carry with it a certain
glamour which helps break down sales resistance. Things of this kind
must be considered by the breeder of purebred livestock. He must find a
sale for his young stock; and it is to his advantage, other things being
equal, to produce the kind of stock which sells most readily. Interchanging
breeding stock from great distances constitutes an economic load on
the purebred industry. In many cases there is no commensurate gain.
The breeding worth of an animal depends upon its genes; and those are
not changed by advertising, although the animal's chance to affect the
whole breed by its genes may be much changed by that.

One of the main business advantages of community breeding is that
lower selling costs can thus be achieved. If a community contains many
herds of one breed of livestock, it may acquire a district reputation in
addition to the individual reputation of the breeders residing in it.
This will attract buyers from a distance because they know they can find
animals which will suit them without heavy traveling expenses in going
from herd to herd. Sometimes a buyer from far away will scarcely bother
to stop in districts unless he feels reasonably sure that within driving
distance of one loading station he can buy a whole carload of animals
which will suit him. This happened frequently in the expansion of the
dairy business in the southwestern states during the decade beginning
about 1920. Buyers coming from that region with orders for an entire
carload of dairy stock would often pass by well-known but isolated
herds in Kansas, Missouri, Iowa, and Illinois to go into counties in Wisconsin
where they thought they could buy a whole carload in two or
three days without traveling far from one shipping station. Community
breeding also makes possible the organizing of co-operative consignment
sales at a low cost for each animal. Often it is scarcely economical
for the ordinary breeder to arrange such a sale since his herd is not
large enough that the costs of advertising and holding such a sale could
be distributed over enough animals to keep the sales cost per head reasonably
low.

Community breeding also makes more effective advertising possible.
Several breeders located near the same place may run a single advertisement
with the names of all signed to it. There need be no business connection
between them except in this advertising. By this means the
public is told that each of them has breeding stock for sale and is also
informed that there are several different flocks or herds from which to
choose, all of them close enough that the buyer may perhaps see them
in a single day with a minimum of time and expense.

As a general rule, the formal organization of a community breeding
enterprise should be kept as simple as possible. Sometimes it is necessary
to have a secretary and a board of directors or executive committee. It is
usually possible to avoid the employment of any salaried officer. Such
expense might increase the overhead expenses enough to offset the business
advantage otherwise inherent in community breeding.

\section*{SUMMARY}

Most of our breeds were formed originally by more or less definite
community breeding. Occasional examples of such community breeding
have occurred in America, but this has not yet become the general
practice in any nationally important breed.
\noclub

Because breeders see their neighbors' animals often and know them
so much better than they do herds at a distance, fewer mistakes are
made in selecting breeding stock from neighboring herds.

Community breeding makes possible the exchange of sires at low
cost and thus preserves the services of the best sires without the necessity
of close inbreeding.

Community breeding can lead naturally to linebreeding which can
be quite effective without the intensity of the inbreeding necessarily
becoming high.

Community breeding gives opportunity for exchanging of experiences
and discussion of problems, thus helping a breeder acquire
knowledge and skill which would take many years if he were operating
in a community where he was the only man with his chosen breed.

Community breeding has many business advantages, among the
most important of which are lower selling costs, more buyers because of
the reputation of the district and the larger number of herds from
which to select, co-operative sales, co-operation in advertising, and the
effective and economical operation of fairs at which local interest may
be keen.
\index{Community breeding|)}