\chapter{Preface}
This book has grown out of sixteen years of teaching animal breeding to college students who already have had
courses in genetics, embryology, anatomy, and physiology of farm animals, herdbook study, history of breeds,
and stock judging. The object of this course is to give the students a clear understanding of the means available
for improving the heredity of farm animals, more especially of what each possible method will or will not do well.

No effort of mine could keep the book entirely free from statistical terms. After all, a breed is a population, and
any attempt at precision in discussing methods of changing its characteristics must necessarily be phrased in terms
of the measurements of populations; that is, in terms of averages and of variability. Complete proofs have been
presented only where those were simple and brief. In other cases I have sought to present only enough to outline the 
argument and to show why it is reasonable. That, of course, involves more facts and formulas than most students will use 
in actual practice but not, I think, more than are necessary to help the student understand why each breeding method he 
might use is effective in doing certain things and practically powerless to do other things.

Animal breeding is a business; and, therefore, economic considerations of the value and availability of time and materials 
loom larger in it than they do in the investigation of a purely scientific problem. The work must go on, and decisions 
must be made in many cases where there is not yet enough evidence to show with certainty what the result will be. The 
scientist, faced with the problem of deciding what is the truth in such a case, might retire to his laboratory and design 
an experiment which in due time would reveal that truth. But the man engaged in the business of animal breeding cannot 
wait for that. Without being entirely certain of what would result from each of the alternatives open to him, he must 
decide whether to cull or keep each animal and whether to mate it in this way or that. Knowing that the odds are two to 
one in favor of one procedure as against another may be highly useful to him as a business man, although the scientist 
may well demand that the odds be higher than the conventional 19 to l before he places much faith in a principle deduced 
from his experimental data. With these needs of the practical breeder in mind, I have sought to state the most probable 
truth concerning questions which may guide his actual decisions, even in cases where genetic knowledge has not yet 
established the limits of that truth as closely as is desirable. Such statements have been labeled with qualifying phrases 
so that the students will be prepared to encounter occasional exceptions to them.

The ideas in this book have been drawn freely from the published works of many persons, I wish to acknowledge especially my indebtedness to Sewall Wright for many published and unpublished ideas upon which I have drawn, and for his friendly counsel.\\
~\\
Ames, Iowa\\
June, 1945\\
\begin{flushright}
	JAY L. LUSH
\end{flushright}