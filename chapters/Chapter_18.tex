\chapter{Breed Type}
\label{cha:Lush_Chapter_18}
\index{Breed purity|(}
\index{Breed type|(}
\index{Deviations from type|(}
\index{``Fancy points''|(}
\index{Percentage of blood|(}
\index{``Pet and fancy stock''|(}

Breed type means the complex of external characteristics which is
typical of a breed or is considered ideal for that breed. The term is often
used in distinguishing one breed from another breed used for much the
same purpose. Many ingredients of breed type are conspicuous details
of conformation and color which have no relation to the economic productivity
of the animals. Examples are the shape of horns in cattle, the
dish of face and size and shape of ear in swine, color of face and shape
of ear in sheep, and color generally. It is those features of breed type
which are the subject of this chapter.

Attention is paid to breed type mainly because it is a ``trademark''
which is some additional evidence that the animal really conforms to
the ideals of the breed. Probably the men who breed purebred animals
average somewhat higher in honesty than men in most other lines of
work, since the foundation of the pedigreed livestock business is the
honesty of the men who sign their names to the pedigrees. Without
general honesty on this point purebred animals could not command the
premium they do. Yet there will always be a few mistakes and frauds.
The existence of a definite breed type, especially if that is a combination
of characters hard to obtain without absolutely pure breeding, is
one check upon errors in registration. If a breed has a type of this kind,
an animal deviating markedly from that type will be regarded with suspicion.
From this point of view the breed type which is the hardest to
attain or which is the most easily upset in crosses is the most highly
prized. This often goes to extremes in the case of ``fancy stock,'' such as
pigeons, rabbits, or dogs, the desired type or standard being kept just
high enough or changed just often enough that only a small proportion
of the breed attain it.

Of course, breed type also is a matter of beauty to the men who have
long been breeding and admiring that breed. But beauty is very much
a subjective matter. Most of us can bring ourselves to think that any
particular type is beautiful if we work with it and study it long enough
and find it profitable. Naturally the breeders of other breeds may not
share our enthusiasm for the supposed beauty of our breed.

Part of the demand for breed type originates more or less unconsciously
with breeders who are enthusiastically steeped in the tradition
of the ancient purity of their breed. It is easy for such men to persuade
themselves that ``the best animals of the \rule[2.25pt]{2.54cm}{0.5pt} breed with the
purest blood are always thus and so,'' and to believe that deviations
from that description indicate impurity of breeding or that something
went wrong with the hereditary process.\footnote{\textit{See} W.
Engeler's interesting account of the theory of ``racial constancy'' and
selection for breed type in European animal breeding writings from 1800 to
1880. Pages 45--58 in ``Neue Forschungen in Tierzucht,'' Bern, 1936.}
\noclub

Insistence upon conformity to breed type is actually harmful only as
it weakens the intensity of selection for economically important points.
This it must do to some extent.

An example of how insistence on breed type may change a breed is
the occurrence of red spots on the faces and red rings around the eyes in
the Hereford breed. Many of the Herefords imported to America carried
these red markings. There was at first no prejudice against this;
and, in fact, certain breeders rather preferred these, which they called
``brown eyed.'' Eventually the tide of favor swung toward faces as white
as possible. Today one sees few purebred Herefords which have complete
red rings around the eyes. In the extreme southwest part of the
United States, Herefords with pure white eyelids are more subject to
cancer of the eyelid than are those with red eyelids. This is not an
important matter, because only a small fraction of those with white
eyelids develop cancer. Moreover, the cancer develops slowly and only
upon the older animals. A ranchman usually has time to cull those
affected and to ship them to market without suffering a complete loss.
This is a minor disadvantage, but many ranchmen wish that the Hereford
had kept its original high frequency of red-eyed cattle. Why did
the Hereford breeders select the white-eyed type when there was no
criticism of the utility of the red-eyed ones? The answer seems to be that
among the very first things to appear in crosses of Herefords with other
cattle were red spots on the face and red rings around the eyes. To many
a cattleman, the presence of red spots on the face or red rings around
the eyes of Hereford cattle indicated impurity of breeding. With this
customer opinion confronting him, it was almost inevitable that the
breeder of purebred Herefords should select for those which had the
most nearly white faces and white eyelids.

Another striking case was the strong preference for yellow color in
honeybees which arose soon after yellow Italian bees began to replace
the black or German bees in the United States. Many beekeepers
inferred that the yellow color was itself the cause of the practical qualities --
gentleness and superior honey-gathering ability -- they wanted.
They began to select for yellow color itself. Charles Dadant found in
Italy some bees darker than most Italian stocks but more productive.
These he could hardly sell to American beekeepers who had by that
time come to believe that the yellower the bee the better. Who knows
how many years the practical progress of beekeeping was retarded by
that color craze!

Such examples are by no means confined to the United States. Even
in Denmark, where such high emphasis is placed on practical utility in
all livestock, the Red Danish cattle are not eligible for prizes at some of
the important shows if they have a large amount of white or any roan
color. The reason is that Shorthorn blood was used many years ago in
some attempts to improve the Red Danish breed by crossing. In time
that came to be regarded as a failure, and every effort was made to cull
from the breed all animals carrying any traces of that Shorthorn outcross.
Who knows how much of the Shorthorn dislike for a dark muzzle
in Britain and the United States today is a similar hangover from the
controversy about the use of the ``Galloway alloy'' in the days of the
Collings? Or how much of the Guernsey dislike (in the United States-little
attention is paid to it in Guernsey) for a dark muzzle stems from a
desire to emphasize the distinction between them and the Jerseys? Similarly,
in Denmark the insistence of the breeders of Landrace swine that
their swine shall have very large and drooping ears seems logically
explainable only on the ground that this is one of the few outward distinctions
between the Landrace and the more or less competitive Yorkshire,
both breeds being bacon hogs, long-bodied, and solid white. In
the tropical parts of Brazil, breeders selecting among pure zebu cattle
try to get them with ears as long as possible. Presumably this is an aftermath
of the extreme competition which prevailed when the zebu cattle
and their grades were first getting a foothold. The grades with the most
zebu blood generally had the longest ears. Hence length of ear, originally
preferred because it indicated a high percentage of zebu blood,
came finally to be considered as itself a sign of higher merit.
\index{``Pet and fancy stock''|)}
\noclub

Some of the things which constitute breed type cannot be fixed.
Laboratory experiments with various piebald races of animals such as
guinea pigs, have shown that a considerable amount of variation in
extent of white spotting is not hereditary at all and would still exist in
a homozygous strain or pure line. There is every reason to think that
the same situation prevails in such spotted breeds as the Holstein-Friesian
or Guernsey or Spotted Poland-China. No amount of selection, no
matter how long continued or with what inbreeding system it was combined,
could ever produce an absolutely uniform breed. This is in
spite of the fact that in these breeds there are modifying genes which
tend to restrict or extend the pigmented areas. Doubtless the same thing
is true with spotting which takes a more definite form as, for example,
the amount of white in the Hereford pattern or the amount and uniformity
of the white belt on Hampshire swine.\footnote{For a plausible genetic
explanation of white belt in swine, sec Olbrycht's article
in \textit{Annals of Eugenics} 11:80--88, 1941} Wherever this is the
situation and the breed type is really unfixable, it is especially regrettable
when otherwise desirable animals are discarded from the breeding
herd on account of failing to conform closely to a rigid standard of
color markings. Not only are their good qualities lost to the breed but,
ironically enough, their discarding does not cause the breed to conform
more closely to the standard breed type than if they had been kept for
breeding purposes.

The matter of breed type is receiving less attention from breeders
today than it has many times in the past. The practical breeder cannot
afford to neglect it altogether wherever it still has some cash value in
the market in which he must dispose of his surplus. He needs to satisfy
his customers as much as he can without losing much real productivity
from his stock. Any more efforts in selecting for breed type than his
customers' demands absolutely force on him detract from his ability to
select for things of practical utility. When a breeder hears that his customers
very much want certain features of breed type, it behooves him
to be skeptical about whether they really will pay him much more for
the animals which have those things. Some statements of this kind are
just sales talk or buyer's talk\footnote{\textit{Proverbs} 20:14.};
others are details in an almost endless and unbalanced catalogue of
all characteristics which ever have been noticed. Many of these details
will have little or no detectable effect upon the amount the buyer really
will pay.
\nowidow
\index{``Fancy points''|)}

\section*{SUMMARY}

Breed type serves as a trademark, which is to some extent additional
evidence of purity of breeding over and above the printed pedigree. It
is not as valid evidence on this point as is sometimes believed.

Some of the elements of breed type probably cannot be `fixed'' so
that they will be perfectly uniform. Effort spent in breeding for those
not only weakens selection for more important things but even fails to
improve breed type.

Breed type becomes positively harmful when so much attention is
paid to it that animals above the average in real usefulness are
discarded because they do not conform to breed type in matters which are
of little or no economic importance. The more points considered in
selection, the less effective can selection be for each of them. Breed type
should be kept in a minor place, but the practical breeder cannot afford
to ignore it altogether if his market places some value on it.
\index{Breed purity|)}
\index{Breed type|)}
\index{Deviations from type|)}

\section*{REFERENCES}

\begin{hangparas}{0.5in}{1}%
Van Riper, Walter. 1932. Aesthetic notions in animal breeding. Quart. Rev. of
Biology, 7:84--92.

Wentworth, E. N. 1926. Breed, show and market standards. Proc. of the Scottish
Cattle Breeding Conference in 1925, pp. 212--18.

Wriedt, Christian. 1930. Heredity in livestock. See especially the chapter on
``Fairs and Fancy Points.''
\end{hangparas}