\chapter{Masculinity and Femininity}
\label{cha:Lush_Chapter_34}
\index{Femininity|(}
\index{Masculinity|(}
\index{Sex reversal|(}

In many writings on stock judging and animal breeding, it is urged
that sires which are masculine in appearance should be chosen and
those which appear somewhat feminine should be avoided. It is likewise
stressed that a feminine appearance is desirable in females. One of the
reasons occasionally advanced for this is that such individuals will be
more prepotent than others. As explained earlier, this is without experimental
support. The belief may have arisen incidentally from other
and better-founded reasons for desiring full development of the secondary
sexual characteristics in breeding animals.

\section*{ACTIVITY OF PRIMARY SEX GLANDS}

The development of the secondary sexual characteristics is controlled
to a large extent by hormones secreted by the primary sex glands
(ovaries or testes) in a normally healthy individual. Variations in the
expression of secondary sexual characteristics may indicate variations in
the activity or state of health of the primary sex glands. Secretion of the
sex hormones is not identical with the activity of the sex glands in producing
sex cells (ova or spermatozoa). Thus, in ridglings \index{Cryptorchidism}(cryptorchids)
the testicle which is retained high in the body cavity rarely produces
functional spermatozoa. It seems, however, to secrete the sex hormone
in almost if not quite normal amounts. Males possessing a cryptorchid
testis but having had the normal testis removed show the secondary
sexual characteristics and behavior of normal males, although they are
rarely able to beget offspring. Among birds, cases have been reported
where individuals which have been functional as females later became
apparent males. Nearly all of these birds when dissected show that the
ovary (the adult female bird normally has only the left ovary instead of
two as mammals have) which had originally been functional when the
individual was a normal female had become diseased by tuberculosis,
cysts, or some other condition, until it had wasted away and was no
longer able to produce the sex hormones. In short, the hen had been
physiologically castrated,\footnote{``Spayed'' is the word more commonly
used in connection with females in animal breeding, but ``castrated''
may be used for either sex in scientific writings.} and the effects
were practically the same as would have occurred had she been castrated
with a knife.

Incidents similar in principle sometimes occur among the mammals.
A familiar example is the \index{``Free-martin''}free-martin (see chapter~\ref{cha:Lush_Chapter_35}),
which is a heifer born twin with a bull. Such heifers are usually barren
and often are quite masculine in appearance. Examination usually shows
that the sex organs are in a rudimentary or abnormal condition. No doubt
some other cases, besides free-martins, where females are quite masculine
in appearance are really cases of poor functioning of the ovaries. Perhaps
some males do not appear normally masculine because their testes are
in some way functioning subnormally.

Thus, one of the reasons which the breeder has for seeking masculinity
in his males and femininity in his females is that such evidence is
some indication of the normal health and functioning of the primary
sex glands. Certainly there are many exceptions to this rule, and probably
it is not worth much attention if the seller will guarantee that the
animal in question is a sure breeder. H. H. Wing tells of a bull which
sired three very good daughters in the Cornell University herd but, on
account of his feminine appearance, was sold before his daughters'
merits became known. It should be added that the physiology of hormone
action is not simple. There are many reactions and complicated
interactions of the hormones from the sex glands with the hormones
from other sources.
\index{Sex reversal|)}

\section*{ABNORMAL DIVISION OF CHROMOSOMES}
\index{Chromosomes|(}
\index{Intersexes|(}

A second mechanism which may cause deviations from normal sex
characteristics is abnormal behavior of the chromosomes. Definite evidence
of this exists for Drosophila and some other laboratory organisms.
It is reasonable to suppose that such behavior would occur occasionally
among the mammals. Sex is not determined simply by whether
the number of \textit{X}-chromosomes present is one or two, but depends
upon a balance between the effects of genes trending toward femaleness
and genes trending toward maleness, most of the former being present on
the \textit{X}-chromosomes and most of the genes trending toward maleness
being scattered on the autosomes. Normally, the presence of one or of
two \textit{X}-chromosomes throws the balance completely in one direction or
completely in the other. If the chromosomes do not divide regularly, as
happens in rare cases, an individual may have a few more or less than
the normal number of chromosomes. This may keep the balance from
turning definitely toward maleness or toward femaleness and may result
in intersexes of various kinds. Some of these intersexes are so extreme as
to be sterile, but less extreme ones are sometimes fertile. In Drosophila
this process is known to have produced occasional individuals which
are more feminine than normal females or more male than normal
males-the so-called ``super females'' and ``super males.'' These arc
sterile. Winge has reported a case in which abnormal division of the
chromosomes in a species of fish resulted in one race becoming homozygous
for the \textit{X}-chromosomes. Another pair of chromosomes, which evidently
was not homozygous for all the genes affecting the expression of
sex, took over the function of normally throwing the balance toward
maleness or toward femaleness in each individual. In fishes, amphibians,
and birds it appears that the balance normally thrown toward
maleness or femaleness by the chromosome mechanism can more easily
be reversed by genes on other chromosomes or by environmental conditions
than is the case among the mammals.

Such chromosomal intersexes as were not sterile would transmit to
some of their progeny the abnormal chromosome balance which caused
them to deviate from the normal expression of sex characteristics. I£ a
similar condition exists among mammals, some forms of intersexual
conditions may be inherited. The use of breeding animals showing
intersexuality might result in increasing the amount of intersexuality in
the next generation. It seems improbable that this occurs often enough
among the mammals to deserve much attention, but it is a possibility.
\index{Chromosomes|)}
\index{Intersexes|)}

\section*{GENES AFFECTING SECONDARY SEXUAL CHARACTERISTICS}

A third cause of variations in masculinity and femininity is the
action of definite genes affecting the expression of the secondary sexual
characteristics. Several of those are known in Drosophila and some,
such as the genes for ``hen feathering'' in poultry, are known in other
animals. No doubt some of these exist in all kinds of animals and also
in such plants as exist in separate sexes (are ``dioecious''). In fact, the
results of abnormal chromosome behavior just discussed are difficult to
explain on any other basis than that there are various genes affecting
the expression of sex differences, a high proportion of the genes operating
toward femaleness being located on the \textit{X}-chromosome while most
of those which operate toward maleness are located on the autosomes.

Further evidence about the existence of such genes comes from race
crosses. These have been most extensively studied by Goldschmidt in
the gypsy moth, Lymantria. Crosses between certain races of these produce
intersexes of various degrees of intersexuality, while others produce
normal individuals. A given cross always behaves in the same manner.
That is, the results are orderly and definite. It seems likely that the
nor~al balance between maleness and femaleness is caused by somewhat
different combinations of genes in different races. In any comparatively
pure race the balance falls definitely in one direction or the other.
When two races which differ in the genes controlling this balance are
crossed, the delicate balance required to throw the mechanism of sex
determination in one direction or the other is likely to be upset. We do
not know how general this is in the animal world, especially among the
mammals. It may be so rare that it scarcely deserves mentioning. Perhaps
the best general evidence on this subject is that from the sex-ratio
in species crosses. Sometimes this is not disturbed at all, but often it is.
Wherever there is a disturbance the heterogametic sex is usually the
more violently affected. Among the common farm animals the mule\index{Mules} is
nearly always sterile, although a few cases of fertile mare mules have
been reported. No cases are on record of a fertile male mule, but in fairness
it should be added that the circumstances are such that fertility in
the male mule would be more rarely detected than in the female. In the
crosses between domestic cattle and the American bison, the few males
produced have all been sterile, but most of the females have been
fertile.

The nature and degree of secondary sexual differences vary from
race to race within the same species. There is a somewhat greater difference
in temperament between bulls and cows of the dairy breeds than
in beef breeds, although a part of this may be a result of differences in
the way they are managed. In some breeds of sheep, horns are a masculine
trait; in others they appear in both sexes; in still others, neither sex
has horns. No breed of sheep normally has horned ewes but hornless
rams. In man, many racial differences occur in the expression of sex
differences. In the races from northwestern Europe and around the
Mediterranean region, the men are generally rather heavily bearded.
Some of the European peoples and most of the peoples of eastern Asia
as well as the original natives of the two Americas are scantily bearded.
The various negro races of Africa differ much among themselves in this
respect. The beard is regarded as a secondary sexual characteristic in
man, but its degree of expression varies from race to race, and that
variation is not accompanied by corresponding variations in masculinity.
In practically all races the height of the men is greater than that of
the women, but the proportion of this difference varies from race to
race.

Specific genes affect the expression of the secondary sexual characteristics,
and many seem to have no direct bearing on reproduction
itself. No doubt many of the differences in masculinity and femininity
which we see or discuss in our breeding selections are the effects of such
genes. While such genes may have no direct physiological importance,
yet so long as the customers desire masculinity in the males and femininity
in the females it will be to the breeder's interest to produce the
kind of animals they want to buy. Probably users of dairy cattle would
be better off if gentleness and meekness had always been sought by
breeders of dairy bulls. Many of the dangers of handling aged bulls
would have been diminished.

\section*{SUMMARY}

Masculinity in males and femininity in females to some extent are
expressions of the functioning of the testes or ovaries in secreting hormones.
Absence of masculinity in males or of femininity in females may
indicate lowered functioning of these glands, which might in some
cases be extreme enough to cause these individuals to be irregular in
breeding or even sterile. The importance of evidences of masculinity
and femininity has often been exaggerated. The most valid reason for
desiring manifestations of normal secondary sex differences is that those
may be a partial guarantee that the animal will be a regular breeder.

Abnormal chromosome distribution may on rare occasions disturb
the balance of gene action which normally determines complete maleness
or complete femaleness. The resulting intersexes, if fertile, may
transmit an abnormal number of chromosomes to some of their offspring,
thereby leading to some inheritance of this intersexuality.

Some genes definitely affect the expression of sex differences without
any detectable effect on the real efficiency of the animal in reproducing
itself. These lead to inherited differences in the expression of
secondary sexual characteristics. The breeder will need to pay some
attention to these differences if his customers do.

\section*{REFERENCES}

\begin{hangparas}{0.5in}{1}%
Craft, W. A. 1938. The sex ratio in mules and other hybrid mammals. Quart. Rev.
of Biology. 13:19--40.

Deakin, Alan; Muir, G. W.; and Smith, A. G. 1935. Hybridization of domestic cattle,
bison, and yak. Tech. Bul. 2, Dept. of Agriculture, Dominion of Canada.

Goldschmidt, Richard. 1934. Lymantria. Bibliographia Genetica, 11:1--186.

Lush, Jay L.; Jones, J. M.; and Dameron, W. H. 1930. The inheritance of cryptorchidism
in goats. Texas Agr. Exp. Sta., Bul. 407.

Warwick, B. L. 1935. Inheritance of the ridgling characteristic in goats. Texas Agr.
Exp. Sta., 48th Annual Report, pp. 35--36.

Wing, Henry H. 1933. The Cornell University dairy herd, 1889 to 1928. Cornell
University Agr. Exp. Sta., Bul. 576.

Winge, 0. 1934. The experimental alteration of sex chromosomes into autosomes
and vice versa, as illustrated by Lebistes. Compt. Rend. Laboratoire Carlsberg,
21:1--50.
\end{hangparas}