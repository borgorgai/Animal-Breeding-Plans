\chapter{The Relation of the Breed Association to Breed Improvement}
\label{cha:relationship-breed-association-breed-improvement}
\index{Breed associations|(}

The activities of the breed associations are intended to maintain the present merit of the breed, to improve the merit of the breed, and to promote the business interests of the members. Some activities serve all three of these purposes, and many serve two of them.

The primary object of breed associations, particularly in countries to which the breed is not native, is to safeguard the 
\index{Breed purity}purity of the breed and to furnish accurate pedigrees to all breeders who desire them. In practically all cases the breed 
association took over the conduct of the herdbook early in the history of the association. Most of the clerical work in 
most associations is used for issuing and checking the accuracy of registrations and transfers. Most of the errors 
discovered in applications for registry or transfer are matters of carelessness or neglect, but close watch is kept for 
fraud, and occasionally a member is expelled for this reason or his registrations are canceled. Because of the possibility
of legal complications and internal dissension, this is not usually done unless the proof is quite conclusive. The accuracy 
of a pedigree depends mainly upon the honesty and carefulness of the man who signs the application for registry. But the 
breed association's policy of rejecting or returning applications not accurate in detail and of investigating within 
reasonable limits cases where fraud is suspected prevents many errors which would come into pedigrees if there were no such 
supervision. Some associations publish the facts when a member is expelled or registrations are canceled because of fraud. 
Others keep it quiet to avoid scandal. Probably the first policy is generally the wiser. In either case the news usually 
spreads fast. Safeguarding the purity of a breed does nothing to increase its merit but does act to some extent as a ratchet
mechanism to maintain whatever special merit the breed already has and to hold any future improvements which may be made in 
its average merit. Preserving the present merit of the breed is particularly important if the breed is an introduced one, 
few in numbers, and surrounded by animals of distinctly different origin.

\index{Shows and fairs|(}
A breed association tries to \textit{improve} the merit of the breed by guiding the ideals which the breeders use when making their 
selections and by making official tests or ratings of the productiveness or conformation of individual animals. The ideals 
of the breeders are influenced by such activities as adopting a score card\index{Score cards}, or other verbal description of the breed ideal, 
or a series of ``true type''\index{``True type'' models or pictures} pictures or models. A different approach to the same goal is through control of judging 
standards at the various fairs. Sometimes that amounts only to advising the fair management upon request whether the breed 
association considers a certain individual competent to judge that breed. In other cases the breed association prepares a 
fairly small list of men considered competent, and its contributions of prize money are contingent upon the fair management's
selecting its judge from that list. Sometimes the association prints circulars or uses advertisements which serve the 
double purpose of promoting wider interest in the breed and showing pictures of animals which are considered nearly ideal 
for that breed. The association sometimes instructs its members as to what is considered important in pedigrees (See 
example in \textit{Holstein-Friesian World}, 33: 300, April 4, 1936).

The general purpose of all these ways of teaching the breeders the official ideals for that breed is that the breeders thus 
informed shall follow those standards when making their own selections and cullings and will thus move the breed average 
nearer to the breed ideal. The breed association's efforts end with presenting the lesson to the breeder. Unless he asks 
for further help from the fieldman or other officer, it is entirely up to him whether he accepts the official ideal and how 
much or how little he uses it when making his selections and cullings.

Official tests of the speed of individual horses were characteristic of the Thoroughbred and the Standardbred from the very 
beginning. In the case of the latter, a certain speed was necessary for registration hence the name of the breed. As long 
ago as 1832 the preface to the list of Thoroughbred horses in Prussia contained the statement: ``Unless herdbooks contain 
production tests\index{Production testing|(}, they will be useless and without interest, since they will contain only names, of which no one knows
anything and which mean nothing.'' (Engeler, 1936). Official testing of dairy cattle began around 1880\footnote{Pages 15 and 
16 of \textit{Holstein-Friesian History}, by Prescott, Price, Wing, and Prescott.} with the Holstein-Friesian breed in the
United States, largely as a result of the work of Solomon Hoxie. (Private production records were being kept at least as 
long ago as the time of Thomas Bates.) For a number of years there were serious doubts as to whether official testing would 
become popular enough to be retained. However, it eventually proved so useful, not only for breed improvement but also as 
an aid in advertising and selling, that no dairy breed association would now try to do without its department of official 
testing. But such official tests are not yet a prerequisite to registration. Whether the breeder will test or not is still 
entirely voluntary with him, except as the demands of a considerable portion of his customers put some economic compulsion 
on him to test.

In the United States, official inspection of whole herds, with a rating of each animal for type, was begun on a voluntary 
plan in 1929 by one of the dairy breeds and has since been adopted by most of the others. This plan was used by only a few 
breeders during the depression of the 1930's but now seems to be spreading more rapidly. Type inspection or scoring, 
sometimes on a compulsory basis, has been practiced much longer in some other countries; for example in The Netherlands or
Switzerland.

Many of the swine associations have required the man who registers a litter to report the number of pigs farrowed and 
raised. Some have published in the herdbooks the number farrowed in each litter. Recently Records of Performance have been 
established by several of the swine breed associations, the Hampshire having been the first. These are based mostly on 
weight of litter weaned at 56 days. They are still voluntary and are just beginning to be tried on an extensive scale. They
seem to be a sound step forward, but their wide adoption will probably depend on how insistently the breeder's customers 
ask for such information when they come to buy boars.

Tests for measuring the practical productiveness of beef cattle are being studied at experiment stations,\footnote{See Montana
Agr. Exp. Sta., Bul. 397; also Minnesota Agr. Exp. Sta., Tech. Bul. 94; also \textit{Empire Jour. of Exp. Agriculture},
8:259--68.} but these have not been adopted by any of the beef breed associations.

There have been some discussions of a Record of Performance for sheep, particularly for the fine-wool breeds. The shearing 
records made in New England a century ago were a kind of crude beginning in that direction. No definite plan is in actual 
operation today.

Although there has been some experimenting with endurance rides for cavalry horses and with pulling tests for draft horses, 
these have not yet been made an official part of breed association activities.

The associations in the United States do not usually keep official lists of the prizes won by individual animals at the 
important shows, although some of them do so in an unofficial way. In many cases the breed paper performs that service.
\index{Production testing|)}

With a very few exceptions --- such as the speed requirement for the Standardbred horse, the type\index{Type classification} inspection given to Brahman cattle which are admitted as foundation stock, and the flock inspections which are a part of some plans of poultry improvement --- the purely voluntary nature of the production-testing and type-rating done under the auspices of the breed associations in the United States is like that in Britain, but is in marked contrast to the compulsory inspections or testing
in some breeds in continental Europe. Those are discussed in more detail in the chapter on selective registration\index{Selective registration}. In considering how far it would be possible or wise for the American associations to go in that direction, there are broad general questions as to how far collective policies and efforts can or should replace or supplement individual freedom
to follow whatever \index{Breed papers|(}breeding policies or use whatever purebred animals one pleases, without regard to whether those would be
approved by a majority of one's fellow breeders or by an official inspector of the association. Besides such questions of 
general principle, there are also some immediately practical questions of expense which might make impossible, in breeds as 
widely scattered as many of those in the United States, procedures which are feasible in lands where the breed is highly 
concentrated in one or a few localities. Most directors of the breed associations in the United States are reasonably eager 
to adopt any new practices or requirements which will increase the merit of their breed faster than at present. They 
usually demand that the new plan shall be tested in actual operation, however, before they commit their association 
definitely to it. They have sometimes tried a plausible scheme only to find that it did not work as well as they had felt 
sure it would, and traces of the confusion and discontent which resulted have remained for years to plague them.

Naturally, many activities of the breed association are directed mainly at promoting the present business interests of the 
members. Examples are the efforts to expand the breed numbers by getting new breeders to establish herds, the promotion or 
management of sales and the correspondence which the secretary's office has with would-be purchasers of breeding stock. 
Most associations prefer to give support and encouragement to sales efforts but not actually to manage the sale themselves
lest the dissatisfactions which inevitably arise about some transactions should result in animosity toward the association 
itself.

The provision of prize money at the more important shows is intended to promote the breed by bringing out larger numbers, 
thus giving more advertisement to the breed. It is also intended to improve the breed by teaching more people the ideal for 
that breed.
\index{Shows and fairs|)}

Many of the larger breeds have one or more papers devoted mainly to promoting the interests of the breed. Most of these 
breed papers are privately owned and managed, but some are owned and operated by the association itself. The contact 
between the association and the breed paper is close and important to both parties, even when the paper is privately owned. 
Most of the activities of the breed paper are devoted to the immediate business interests of the members --- advertisements 
and news of sales, merchandise for breeder's needs, etc. --- but some of these papers carry articles and information helpful 
in improving the merit of the breed, but not otherwise a matter of financial profit to anyone. Besides the breed papers, 
most associations print leaflets, buy advertising space in other magazines which reach stockmen, and make occasional use of 
the radio as part of their regular efforts at breed promotion.

The activities of the fieldmen or breed extension service are in considerable part devoted to expanding the numbers of the 
breed and helping new breeders get started. The fieldmen work also with boys' and girls' clubs and help established 
breeders with their problems. 

The forms of government of the breed associations vary widely. Usually the policies are determined by an unsalaried board 
of directors, preferably with overlapping terms and only one-third elected each year to prevent erratic changes in 
policies. The executive work of carrying out those policies is administered by a paid secretary. Sometimes there is a fixed 
number of shares of stock, as in most industrial corporations, and one can become a stockholder only by buying a share from 
someone else. More often the number of shares is not limited but any breeder approved by the board of directors may become 
a member by paying a fixed (usually small) sum for a non-transferable membership. Sometimes all members present at the 
annual meeting can vote. In other associations members not present can vote by proxy. Other associations, especially those 
with large financial reserves, have more or less elaborate systems of delegates or representatives chosen by districts, to 
ensure proportional representation and to avoid abuse of the proxy system. The form of government is important insofar as 
it may promote or endanger stability in conducting the association's activities, make it easy or difficult for the board of 
directors to see that the secretary carries out the general policies they wish, and make it easy or difficult for a minority
to seize and hold control. Some of the cases where there are two or more associations for the same breed had their origin 
in intense dissatisfaction on the part of a group which was not in control, either because it was a minority or because 
control had been seized by another group which could not be ousted with the existing machinery for governing that 
association. At one time there were seven separate organizations recording Poland China pedigrees. The number was not finally
reduced to one until 1946.\index{Breed papers|)}

The current problems of the breed associations are numerous. Financial problems have been acute with many associations 
since 1920. Nearly all of the swine and sheep associations and some of the cattle and horse associations have suspended 
publication of their herdbooks. What the final substitute for the printed herdbook will be is not yet evident. In some 
breeds there are still two or more registration associations, with some duplication of operating expenses.

An innovation which is still in the experimental stage in the United States, although it was used by breeders of Thoroughbreds a century ago and advocated by Thomas Bates as a means of overcoming what he thought were defects in the conduct of the Coates Herdbook for Shorthorns, is the filing of \index{Birth certificates} birth certificates which act as tentative registrations. They keep the date of birth and the records of parentage straight but permit the owner to wait until the animal is mature to complete the registration. This is a common practice in those European associations which require inspection for conformation. Since such inspection must wait until the animal approaches maturity, records of parentage might become lost or incorrect in the interval if this precaution of filing a birth certificate were not taken. The increasing percentages which are incorrect when registration is delayed is illustrated by the following data\footnote{\textit{Holstein-Friesian World}, June 21, 1941, page 708.} on 19,172 Holstein-Friesian applications received during a six-weeks period.

\begin{table}[htbp]
	\centering
	\begin{tabular}{cc}
		\textsc{Age} & \textsc{Percentage Incorrect} \\
		Under 2 months  & 13 \\
		2--6 months & 21 \\
		6--12 months & 25 \\
		12--18 months & 33 \\
		18--24 months & 44 \\
		Over 24 months & 53 \\
	\end{tabular}
\end{table}

The Jersey association has recently experimented with selective registry and a pedigree rating system
(the ``star bull'' plan) for bulls. Several dairy associations are trying plans for calling attention officially
to unusually meritorious proved sires; for example, the Ayrshire approved sire plan, and the Holstein-Friesian
publication of sire indexes.

Herdbooks, as printed in the past, permit the tracing of pedigrees in only one direction; that is, one can learn from the 
herdbook what an animal's ancestors were but cannot find what offspring it had. Often a full list of an animal's offspring 
is more valuable than all that could be learned by studying its pedigree. Most breed associations maintain office records 
which will permit making such lists (at least from females), but those are not published. The nearest approach to published
lists of this kind is in the reports of official testing in the dairy breeds where the tested offspring of a given sire or 
dam may quickly be found. Usually little or nothing is published about the offspring which were not tested officially, 
although the Ayrshire association attempts to learn what became of each untested daughter of the bulls in its ``approved 
sire plan.''

In the United States the breed associations receive no direct financial support from the state or national governments. 
Correspondingly there is no governmental control or supervision of the associations or their activities, beyond whatever 
legal regulations apply in general to all nonprofit corporations or associations. However, the representatives of the 
United States Department of Agriculture and of the state agricultural colleges cooperate in many ways with the breed 
associations in activities which are expected to improve the practical merits of the breeds or to benefit the buyers of 
purebred sires. Examples are the supervision of official production tests for the dairy breeds, helping in the management 
and promotion of livestock shows, conducting purebred sire campaigns, promoting the use of proven sires, etc. In some 
countries the governments extend considerable financial aid and correspondingly exert some control over association 
policies. The details of such arrangements vary widely. Examples are Switzerland, The Netherlands, and (especially since 
1936) Germany. In other countries, such as Canada and the Union of South Africa, the government cooperates in supervising 
registration and printing the herdbooks but does not participate in or control other activities of the associations. In yet
other countries, such as Denmark and Argentina, the herdbooks are conducted by farmers cooperatives or by a ``Rural 
Society,'' and the breed association either does not exist or is an advisory and promotional body.
 
Most of the breed \textit{improvement} has to be done by the breeder himself. The association stands ready to help and 
advise him, but it does not select the animals he shall use or decide which he shall cull, except in the comparatively few 
cases where animals are barred from registry because they possess some undesired characteristic. Nor does the association
decide which males shall be mated to which females. The actual selections and the choice of a mating system are left almost 
entirely to the individual breeder to do as he sees fit, provided his animals are purebred and the correctness of their 
pedigrees is unchallenged. Breed associations must remain more or less responsive to the opinions of the majority of the 
breeders and therefore cannot be expected to do much pioneering or testing of new and unpopular ideas. This will have to be 
done by venturesome breeders or by public research institutions. Practical experience indicates that breed associations are 
necessary for breed improvement, since practically every breed which has persisted long has soon developed an association 
to look after its interests. Yet the association's part is on the whole the conservative one of helping hold whatever 
average merit the breed has already attained and acquainting the beginner and the public with what is considered ideal by 
most breeders of that breed. Most of the aggressive positive work toward improving the merit of the breed to still higher 
levels will have to be done by individual breeders who are unusually able, energetic, persevering, or lucky.
\index{Breed associations|)}

\section*{REFERENCES}

\begin{hangparas}{0.5in}{1}%
Anonymous. 1930. ``Notice-Registration Cancelled.'' Holstein-Friesian World,
27 (No. 24):1177.

Advanced Registry Office. 1936. Pedigrees. Holstein-Friesian World, 33:300.

Engeler, W. 1931. Studies on the development and situation of pedigree registering
in the cattle-breeding industry. 92 pp. Internat. Inst. of Agr., Rome.

---. 1936. Die Entwicklung des Herdebuchwesens unter dem Einfluss der Lehren von der Vererbung
und Z\"{u}chtung bei den landwirtschaftlichen Haustieren. (In) Neue Forschungen in Tierzucht. Bern.

Peterson, Guy A. 1919. Swedish herdbook registration. Hoard 's Dairyman, 74:62.

Plumb, Charles S. 1930. Registry books on farm animals.
 
Van den Bosch, I.G.J. 1930. The Holstein industry in America. The scoring system in the Netherlands. 
Holstein-Friesian World, 27:283 and 337.

Wilson, C. V. 1925. A study of the breeding records of a group of Shorthorn cows. West Virginia Agr. Exp. Sta., Bul. 198.

Early volumes of the herdbooks of the breeds in which you are interested. Read especially the minutes 
of some of the early business meetings.
\end{hangparas}