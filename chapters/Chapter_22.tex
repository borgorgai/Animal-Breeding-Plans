\chapter{Prepotency}
\label{cha:Lush_Chapter_22}
\index{Prepotency|(}

Prepotency is the ability of an animal to impress characteristics
upon its offspring to such an extent that they resemble that parent or
each other more closely than is usual. If the offspring are all unusually
like this parent, they will naturally tend to be unusually like each other.
Writings on animal breeding are full of references to prepotency. Many
of those no doubt exaggerate the supposed amount of prepotency
beyond the actual facts. Differences in prepotency do exist, however,
and are sometimes large enough to be practically important.\footnote{Wentworth,
E. N. 1926. Prepotence in character transmission. Proc. Scottish
Cattle Breeding Conference for 1925, pp. 146--63.} Purebreds
are usually more prepotent than crossbreds or grades. An animal may
be prepotent for undesirable as well as for desirable characteristics, but
naturally in breeders' discussions prepotency for desirable characteristics
is mentioned more often.

``Potent'' and ``impotent'' in animal breeding usage refer to the
ability or inability of the animal to reproduce or even to copulate normally.
These terms do not refer to the merit of the offspring.

\section*{GENETIC BASIS OF PREPOTENCY}

Differences in prepotency depend mainly upon dominance and
\index{Homozygosis}homozygosis. In some cases a part may be played by linkage and
epistasis.

The most important cause of differences in prepotency is the degree
of homozygosis in the animals concerned. A perfectly homozygous animal
could produce only one kind of gamete. All its offspring would
receive exactly the same genes from it. Any genetic differences between
those offspring would depend entirely on their having received different
things from their other parents. On the other hand, an animal heterozygous
for \textit{n} pairs of genes could produce $2^n$ different kinds of gametes.
This permits its offspring to differ genetically, not only in what they
received from their other parent, but also in what they received from
the common parent.

\index{Dominance|(}Dominance is the other important cause of differences in prepotency.
Every offspring which receives a dominant gene will show the
effect of that gene. If the gene is completely dominant and the parent is
homozygous for it, then all of the offspring will appear exactly alike for
the effect of that gene, regardless of the inheritance they received from
the other parent. When a parent having many dominant genes is also
highly homozygous, its prepotency is maximum.

A breed which has several conspicuous dominant traits will appear
to be prepotent in crosses with other breeds. This does not mean that it
will be prepotent in other characteristics. For example, crossing of
Herefords with Aberdeen-Angus ordinarily produces white-faced,
black animals without horns. In this case the Angus breed is prepotent
for body color and for the absence of horns, but the Hereford breed is
prepotent for the white face. Body color and the presence or absence of
horns are conspicuous characteristics. One who does not examine the
animals carefully might infer that prepotency is a general characteristic
of the animal as a whole. Probably most statements that a certain animal
transmitted all of its qualities uniformly to all of its offspring are
based on careless observation of an animal which was homozygous for
one or a few conspicuous dominant traits. To the man unfamiliar with
Aberdeen-Angus cattle the mere fact that a group of cattle are hornless
and black would make them seem impressively alike to him. But the
man familiar with black-polled cattle would be looking for other things
and would not be much impressed by this.
\index{Dominance|)}

Linkage\index{Linkage} has the general effect of making most of the offspring of an
individual fall into a smaller number of classes than if there were no
linkage. If there has been considerable selection among the offspring
of that animal, many having been discarded before we see them, linkage
may thus give us the impression of more prepotency than we would
have observed if all the genes had been segregating independently.

\index{Epistatic effects}
Epistasis may sometimes add something to apparent prepotency.
Occasionally a sire will be homozygous for one or more genes which,
when brought together with genes which many of his mates have, will
produce conspicuous results, although the genes may have little apparent
effect when the full combination is not present. As a result the offspring
from such a sire will be unusually like each other and yet
distinctly different from either their sire or their dam. It is not certain
whether this happens often, but it is a possibility and is the most plausible
genetic explanation for some cases reported.

\section*{THE MEASUREMENT OF PREPOTENCY}

After the offspring are produced, there are two ways of measuring
prepotency. The first is to measure directly the resemblance of this
animal and its offspring, as compared with the usual resemblance of parent
and offspring. The second is to measure how closely the offspring of
this sire resemble each other, as compared with the usual resemblance
between half brothers and sisters. In the first method any permanent
effect produced in the parent by environment, dominance, or epistasis
will appear in every comparison of that parent with each of its offspring.
Therefore, the second method is generally to be preferred.
Moreover, the second method is the only one available in cases such as
measuring a dairy bull's prepotency for milk and fat production, since
he cannot himself express those traits. A weakness of the second method
is that the offspring are more likely to have been exposed to the same
peculiarities of environment than parent and offspring are. Thus if one
bull's daughters freshened in a poorly managed herd and a second bull's
daughters all freshened in a well-managed herd, a breeder knowing
only the records and not about the difference in management is likely
to conclude mistakenly that the first bull was prepotent for low production
and the second bull was prepotent for high production. Also, the
second method will give a high figure for prepotency in those cases
where the offspring resemble each other closely but are distinctly different
from either parent, as might sometimes happen if there were much
epistasis in a particular mating. Some breeders would not like to call
such a sire prepotent, since the offspring do not resemble him even
though they are unusually uniform.

\index{Inbreeding|(}
Prepotency has its limits. In the absence of dominance and epistasis,
the most prepotent sire in the world when mated to random-bred
females cannot do more than make his offspring, whim are half sibs,
resemble each other as closely as ordinary full brothers and sisters from
random-bred parents would. The relationship between half brothers
which are not themselves inbred is $\dfrac{1 + F}{4}$, where \textit{F} is the
inbreeding coefficient of their common parent. This relationship is only one-fourth
if the common parent is not inbred, but approaches one-half as \textit{F}
approaches 1.0. Now, if the genes of the common sire were all dominant
in addition to being perfectly homozygous, we might have the appearance
of still greater prepotency than this. The general effect of epistasis
would be lower prepotency, since not all of the dams to which the sire
was mated would have the genes necessary to nick well with those the
sire carried. In exceptional cases epistasis might increase rather than
decrease prepotency.

\section*{THE BREEDER'S CONTROL OVER PREPOTENCY}

Dominance and epistasis result from the physiology and chemistry
of the genes in their reactions with each other and with the environment
in the growth of the individual. The breeder can do little or
nothing to change them. Linkage is likewise not subject to the breeder's
control.

The breeder's control over prepotency is limited to changes he can
make in the homozygosity of his stock. For all practical purposes he
changes homozygosity little except by inbreeding. The more highly
inbred an individual is, the more apt it is to be homozygous for an
unusual number of genes. The inbreeding coefficient is the best estimate
which can be made of an individual's prepotency before that
individual has actual offspring, by which its prepotency can be measured.
Prepotency can be increased only a very little by the practice of
mating like to like without inbreeding. The resemblance between parents
and offspring is much increased by mating like to like, but when
animals bred in this way are mated to unrelated or random individuals,
they show only a little more prepotency than if they themselves had
been random bred.

The increased prepotency of inbred animals has been known at least
since Bakewell's time, but breeders do not generally pay much money
for it in the sale ring. The inbred animal is usually less apt to be prepotent
for its poor traits than for its good ones, but that is not always
recognized. The undesired traits are more often recessive than dominant.
Therefore, they are not apt to appear in the offspring of the
inbred individual when it is mated to unrelated animals which do not
show those undesired traits.
\index{Inbreeding|)}

\section*{MYTHS ABOUT PREPOTENCY}

Prepotency is not transmissible from parent to offspring as other
characteristics may be, except insofar as it depends on dominance. No
matter how homozygous a parent is, it cannot transmit that homozygosis
to its offspring. Its offspring can be homozygous only if they
receive the same genes from both parents. A high degree of homozygosis
can be attained only by many generations of inbreeding but can be
destroyed by a single generation of outbreeding. We sometimes read in
animal breeding history of cases where there was an unbroken line of
succession of noted sires from father to son and to grandson, and so on.
The Baron's Pride line in Clydesdales is an example. Often the history
of the case is accompanied by the inference or perhaps the outright
statement that this sire transmitted prepotency to his sons and his
grandsons. What really happens in such cases is that there is much selection
in each generation, and to some extent each sire's mates are selected
to be like him. As we look back on the breeding history, we note
that one sire was more popular and successful than any other of his
generation, just as his sire and grandsire were; but we do not notice the
large number of half brothers which were discarded in each generation
while finding the leading sire of that generation. If a sire is thought to
be better than any of his contemporaries, he is likely to be bred to better
mates than they are, and more of his sons are likely to be tried as sires in
prominent herds. It is not necessary to invoke prepotency to explain
why the most prominent sire of one generation is sometimes the son of
the most prominent sire of the preceding generation.

Nothing which is known of the mechanism of heredity justifies the
belief that masculinity\index{Masculinity} in a male or \index{Femininity}femininity in a female indicates
prepotency. Those traits are desirable to the extent that they indicate
normal sex instincts and normal health of the sex glands but there is
no reason for thinking that they indicate prepotency.

\section*{SUMMARY}

Prepotency is the ability of an animal to make its offspring resemble
that parent and each other more closely than is usual.

The genetic basis for prepotency is the degree of homozygosity of
the animal and whether its genes are prevailingly dominant or recessive.
To a small extent linkage and epistasis may play some part.

Almost the only control the breeder has over prepotency is the
extent to which he builds homozygosis into his animals by inbreeding.

Prepotency is not transmissible from parent to offspring except insofar
as it depends on dominance. Masculinity or femininity in appearance
probably has nothing to do with prepotency, although it may be
desirable as an indication of normal ability to reproduce.
\index{Prepotency|)}