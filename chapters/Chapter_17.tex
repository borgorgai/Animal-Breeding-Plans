\chapter{Type and Production Records}
\label{cha:Lush_Chapter_17}
\index{Type and production|(}

Breeders pay attention to outward conformation in making their
selections for two reasons. In the first place, they may want a certain
type because it has a market value. If a market demand exists for a certain
type, the breeder may care little whether that type really will furnish
to his customers the maximum profit or other satisfaction. The fact
that they want it and are willing to pay for it is the thing of immediate
practical importance to him. In the second place, breeders may believe
that type and productiveness are closely enough correlated that if one
selects for type he will get at least part of the productiveness he wants.

In many cases, especially among meat animals ready for market, a
certain conformation not merely indicates production but actually
comes close to being production, since the desired production is largely
a matter of sizes and proportions of various parts. At the other extreme
are cases where the desired production depends far more on the quality
and rates of physiological processes than it does on the sizes and shapes
of organs or parts which can be judged on the live animal. The closeness
of the correlation between type and production may be of any
degree ranging from almost perfect in such a case as the width of the
loin or thickness of the round on fat steers, through such relations as
may possibly exist between width of head and width of body, to correlations
which are practically zero. An example of a correlation which was
once thought to be high but has since been found to be practically zero
is the relation which a half century ago was widely supposed to exist
between the escutcheon of a cow and her producing ability.

Reliance on type as a means of estimating productive ability may
be necessary when reliable records of production are not available.
Production records on most animals come slowly and expensively.
Sometimes, as in poultry and dairy cattle, they are not available on both
sexes. Even where production records can be fairly simple and complete,
as in the case of cows and hens, it is still true that many purebred
animals do not have their production recorded. The situation is still
less satisfactory among meat and work animals, where productivity is
not easily nor completely measured. A breeder often has an opportunity
to buy an animal on which no production record has been made, or he
may have to sell some of his young stock before they are old enough
to have production records. Such a breeder, even though he has more
faith in production records as indicating an animal's productivity than
in conformation as a similar index, none the less wants to make as much
use as he can of the animal's conformation in estimating its probable
productivity.

Type has some sale value in all classes of livestock. In extreme cases
beauty may be the main object. This is often encountered in ``pet and
fancy stock,'' such as rabbits, dogs, pigeons, and guinea pigs, and is a
prominent feature of some of the larger livestock such as saddle and
coach horses. If the breeder's customers center their demand on type, he
is of course interested mainly in that, and in productiveness only in
that his animals should remain healthy and fertile. To appear healthy
is, in most cases, an important part of the breed ideal for type also. If
his customers are looking for productiveness regardless of beauty, the
breeder is interested in type only as it may help him get that productiveness
more surely and quickly than if he did not pay attention to
type.

The stockman usually wants lifelong productiveness in each animal
rather than a maximum single record from each, although the advertising
value of an extremely high record may sometimes mean more financially
to a breeder than a higher average record\index{Lifetime Averages} which does not become
phenomenal in any one year. A single production record is not a perfect
index of such life productivity, as was emphasized in chapter \ref{cha:Lush_Chapter_13}. The
question of how much attention to pay to type and how much to pay to
production records in selecting for lifetime productivity is, therefore, a
question of comparing and combining two indicators, neither of which
is perfectly accurate. Yet it would be a rare coincidence if the usefulness
of the two happened to be exactly equal. The principles of estimating
an unknown quantity from two known quantities which are partly
correlated with it are such that a slightly more accurate estimate may
usually be made by using both the known quantities than by using the
more accurate of them alone, although the best proportion in which to
weight the two is much affected by their relation to each other.

Each intermediate step weakens the effectiveness of selection. Thus,
if we want to select for quality \textit{x}, and it is correlated
imperfectly with \textit{w}, we will not come so close to getting
\textit{x} by selecting \textit{w} as we would if we could select
\textit{x} directly.\footnote{As a numerical example, Rasmusson has
shown (1930, \textit{Nordisk Jordbrugsforskning}, pp. 247--55) that
even if the correlation between \textit{x} and \textit{w} is .8, one
wishing to select a certain number of individuals which excel the
population average in \textit{x} by at least one standard deviation
would need to examine 7.8 times as many if he selects them indirectly
by looking for \textit{w} as he would if he could select them directly
by examining them for \textit{x}.} If \textit{x} cannot be observed
directly but is rather closely correlated with \textit{y} and not so
closely with \textit{w}, we would come nearer getting the value of
\textit{x} we want by selecting for \textit{y} than by selecting for
\textit{w}. Unless \textit{w}'s correlation with \textit{x} is
altogether due to \textit{w}'s correlation with \textit{y}, we would
come still nearer to getting the desired value of \textit{x} by
selecting for both \textit{y} and \textit{w}; but the proper amount of
attention to be given to \textit{y} or to \textit{w} would depend upon
how closely they were correlated with \textit{x} and with each other.
The guiding principle on this subject is that every needless intermediate
step in selection should be avoided as far as possible but that a little
is usually gained by paying some attention to other things besides the one
which is most closely correlated with productivity. The very real danger
in that is that one will pay so much attention to these minor things that
he cannot pay enough attention to more important things which are more
closely correlated with lifelong productivity.

\section*{THE CORRELATION BETWEEN TYPE AND PRODUCTION}

In most actual studies of the correlation between type and production
the correlation between one estimate of type or conformation and
one production record of each individual in the population studied was
measured. A few samples of those are mentioned in the following paragraphs.

When official testing began in the Jersey breed, certain judges
inspected and scored the cows admitted to the Register of Merit.
Gowen studied these data to see what correlation existed between the
scores and the production records. Most of the correlations between the
scores for each individual point of conformation and the actual production
records of those cows were of about the magnitude of $- .07$ to $+ .19$.
The correlation between the production record and the total
score of the cow ran somewhat higher, since it took into account all of
the points scored. When the study was confined to the scores turned in
by the nine judges (out of the nineteen recognized ones) whose scores
most closely agreed with the milk yield of the cows, the correlation
between their total score and the milk yield was $+ .38$. While this is a
real correlation, it was obtained only after discarding the scores of half
of these men who were believed by the association to be competent to
score the cows.

In similar studies on early Holstein-Friesian records Gowen used
measurements made on some of the first officially tested cows.
Table~\ref{tbl:Lush_Table_14} presents the correlation coefficients
he found between the seven-day milk yield and various body measurements
and weights.

The maximum correlation between yield and any body measurement
was .36. The correlations found were real and of some use in
selecting the high producers, but they were by no means as high as the
correlations between different records made by the same cow. A considerable
part of these correlations with measurements resulted from differences
in general size. Within a breed the largest cows tend to be the
heaviest producers and naturally tend also to have the largest measurements.

Engeler's study\footnote{1933. Die Ergebnisse statistischer Auswertung
40-j\"ahriger Herdebuch-Aufzeichnungen beim schweizerischen Braunvieh.
Bern: Verbandsdruckerei A. G.} of the yields of 455 Brown Swiss cows and their
scores when they were inspected for registration showed a correlation
of only $+ .04$. In another study 3 of 138 cows\footnote{Schweizerische Landw.
Monatshefte 19, No. 6, 1941.} in one herd he found a correlation of $+ .32$
between milk yield and score.

\begin{table}[htbp]
	\centering
	\caption{\textsc{Correlation Coefficients Between Seven-Day Milk Yields and Certain Aspects
of Conformation, Age Being Constant (After Gowen)}}
	\label{tbl:Lush_Table_14}
	\begin{tabular}{lc}
		\hline
		\hline
		 													& Correlation \\
		Characters Correlated								& Coefficient \\
		\hline
		365-day milk yields in different lactations			& .66 \\
		7-day with 365-day milk yield (same lactation)		& .60 \\
		7-day with 365-day milk yield (different lactation)	& .46 \\
		Weight with 7-day milk yield						& .42 \\
		Body length with 7-day milk yield					& .36 \\
		Body girth with 7-day milk yield					& .25 \\
		Body width with 7-day milk yield					& .28 \\
		Hip height with 7-day milk yield					& .24 \\
		Shoulder height with 7-day milk yield				& .22 \\
		Rump length with 7-day milk yield					& .18 \\
		Thur! width with 7-day milk yield					& .01 \\
		\hline
	\end{tabular}
\end{table}

In dairy cattle there have been several studies of show-ring placings
and production records, where both were known. Because only a small
range of types was included -- no really poor types would be among
those for which the placings were recorded -- such studies throw little
light on the correlation between type and production. They demonstrate
that show animals can produce well, but so might many others if
tested under the same circumstances.

A more suitable basis for studying the correlation between type and
production is in such data as the Holstein-Friesian Herd Classification.
Table~\ref{tbl:Lush_Table_15} shows a summary of such data.\footnote{As
of October, 1942. The records are on a thrice-a-day mature basis and include
the first 365 days of the lactation.} The figures show that type
and production do tend to go together in this population. On the average
the fat production increased 24.6 pounds with each grade the cow
was higher in the type classification\index{Type classification}. The correlation between the two
is a little less than $+ .2$. That leaves plenty of opportunity for very
high producers occasionally to be of poor type and for some animals of
high type to be poor producers. The correlation in the general population
of dairy cattle is probably a little higher than this, since, in the
herds submitted for classification, most of the cows thought by their
owners to be ``Fair'' or ``Poor'' in type already would have been culled
unless their production was unusually good. On the other hand, part of
the apparent correlation may have come about because large size gives
an advantage, both in classification and in production. Also, the classification
may have been influenced in some cases by knowledge which the
classifying officer had of the cow's prior production.

\begin{table}[htbp]
	\centering
	\caption{\textsc{Average Herd Test Records of Holstein-Friesian Cows as Classified for Type
Under the Herd Classification Plan}}
	\label{tbl:Lush_Table_15}
	\begin{tabular}{L{2.5cm}|R{3cm}|R{1.75cm}|R{1.75cm}}
		\hline
		\hline
		Type						& 					& \multicolumn{2}{c}{Average Production} \\
		\cline{3-4}
		Classification				& Number of Cows	& Milk		& Fat \\
		\hline
		Excellent					& 261				& 17,215	& 601 \\
		Very good					& 1,377				& 15,988	& 554 \\
		Good plus					& 2,213				& 15,754	& 544 \\
		Good						& 2,138				& 14,960	& 514 \\
		Fair						& 426				& 14,316	& 488 \\
		Poor						& 25				& 12,612	& 431 \\
		\hline
	\end{tabular}
\end{table}

Table~\ref{tbl:Lush_Table_16} shows the summary of such data on Jersey cows
to April, 1946 on a twice-a-day 305 day basis. The regression of fat production
on type -- the average increase in fat production with each increase of
one grade in type classification -- was 12.8 pounds which is only a little
less than the 24.6 from the Holstein data when allowance is made for
lactation length and times milked per day.

\begin{table}[htbp]
	\centering
	\caption{\textsc{Average Herd Test Records of Jersey Cows Which Were Also
Classified Officially for Type}}
	\label{tbl:Lush_Table_16}
	\begin{tabular}{L{3cm}|R{3cm}|R{3cm}}
		\hline
		\hline
		Type Classification			& Number of Cows	& Average Fat Production \\
		\hline
		Excellent					& 801				& 483 \\
		Very good					& 4,213				& 460 \\
		Good plus					& 6,060				& 448 \\
		Good						& 2,700				& 434 \\
		Fair						& 369				& 420 \\
		\hline
	\end{tabular}
\end{table}

Engeler's conclusion from studies in Switzerland is: ``Form, production,
and health are not so closely related that they can be substituted
for each other as bases for selection. These three characteristics are to a
high degree independent of each other and to a high degree are transmitted
independently to the offspring. The goal of selection consists
in preferring those animals which to the fullest extent possess all three
characteristics, phenotypically as well as genotypically.''

On poultry several studies have been made of the relation between
body measurements and production or of the success which a man
actually attains in attempting to cull out the poorest producers from
each lot, dividing them into two groups and noting their production
immediately after the culling. In the culling demonstrations the apparently
high success is largely the result of immediately preceding conditions,
whereby the man doing the culling is able to identify those which
are laying at that time. Since the test pens are not kept for a full year
afterward, he seems to have been remarkably successful in picking out
the poor producers. It is somewhat the same situation as if one were to
go to a dairy farm and rank the cows according to his estimate of their
milk-producing ability. If his rankings were then compared with the
cows' actual productions in the following week or two weeks, he would
seem to have been remarkably successful; but much of this success
would be due to the fact that he could tell which cows were dry and
which were recently fresh at that particular time! The available evidence
makes it seem doubtful that the relation between productiveness
and conformation is really any closer in poultry than it is in dairy cattle.

Studies of the relation between weight, conformation, and pulling
ability in draft horses have shown (Rhoad) correlations of about the
magnitude of $+ .23$ to $+ .35$ between different measurements and ability
to pull and + .45 between weight and ability to pull. However,
nearly all of the correlation between measurements and ability to pull
seemed to be an indirect result of general size, because the relation
between these measurements and pulling ability within groups of
horses which were all of the same weight was practically zero. In a similar
study Brandt found that weight, age, and five measurements had a
multiple correlation of .47 with maximum pull. Most of this seemed to
depend on weight and height.

Studies of individual beef cattle have been less frequent because
there is no one measure which comes as near expressing their real productivity
as the actual production records of dairy cattle, poultry, and
draft horses do in those cases. The studies which have been made show
moderate or low correlations except for anatomical traits, such as fullness
of round, which can be estimated closely when judged in the live
animal before slaughter. There have been many studies of commercial
grades of groups of feeder steers as related to their subsequent performance
in the feedlot. Usually the steers of the lower grades gain about as
rapidly as those in the higher grades, but they sell at a lower price.
Whether they are equally profitable to the feeder depends mostly on
whether he can buy them at a low enough price. Because of the lower
sale price, the lower commercial grades are less profitable to the
man who breeds them.

Studies of type and production in sheep have been more numerous
but have dealt mostly with the production of wool or with the feeding
qualities of groups of lambs from the crossing of various breeds. Questions
of type are often the subject of discussion among breeders of Merino
and Rambouillet sheep\footnote{See Texas Bul. 657. Also Michigan Quart.
Bul. 26:31--33.} and Angora goats.

Several experiment stations have conducted swine type tests which
have shown differences in the rates of gain and in the kind of meat produced
by swine of types varying from ``very chuffy'' to ``extremely
rangy.'' Studies of the ability of men to predict which pigs would make
maximum gains have generally shown correlations of something
around $+ .4$ to $+ .7$ between the estimate and the actual individual
gain. These high correlations (compared with much lower ones on
steers) are generally reduced to somewhere near the level of $+ .15$
to $+ .30$ when corrected for differences in initial weight.
Greve\footnote{Zeit. f. Z\"ucht., Reihe B., 46:91, 1938.} studied
eight different measurements on 205 sows of the Hoya breed near Hanover
in Germany and concluded: ``All the results show that it is not
possible by using body measurement to find which sows have high
breeding productivity.''

Summarizing the actual evidence on the correlation between individual
type and individual production, there is no complete analysis of
the problem in any class of livestock; but the nearest approach to that
has been in dairy cattle or poultry. In general, these studies have established
the existence of correlations between type and production; but
such correlations are generally much lower than the correlation
between one production record and another production record of the
same animal. The conclusion seems inevitable that if one is interested
mainly in production he should pay much more attention to production
records than to estimates of type, although it does not follow that
he could afford to neglect type altogether. An actual production record
is not quite a perfect indication of the lifelong production which is.
actually wanted. The emphasis placed by many men on outward evidences
of health and ``constitution'' may have some justification in the relation
of those to lifelong productivity. Many of the breeder's decisions
must be made before he can know from actual experience what
that animal's lifelong production will really be. Where type has any
direct relation to lifelong productivity, the latter can be predicted more
accurately by taking into account both type and production records
than by using either alone, but the one which is least closely related to
lifelong production should be given much less emphasis.

\section*{SUMMARY}

A breeder may pay attention to type merely because it will add to
the market value of the animals he expects to produce. Aesthetic considerations
have much to do with the commercial value of some kinds of
animals.

A breeder may also pay attention to type because he believes it to be
a useful indicator of the lifelong productivity of the animal. Such indicators,
even though not as reliable as production records, would be
useful under a variety of circumstances actually encountered, especially
among animals on which production records are lacking.

What is wanted is lifelong productivity. A single production record
is not quite the same thing, although in most cases it is a more accurate
indicator of lifelong productivity than is the individual's type, if either
were to be considered alone.

Where type is related directly to productivity, a more accurate estimate
of productivity can be made by taking type and the available
production records both into account than by considering either alone,
but there is danger of paying too much attention to the less accurate of
the two indicators.

The practical problem confronting the breeder is to give the proper
amount of attention to each. If he pays too much attention to type, the
selection he can practice for production is automatically made less
intense.

\section*{REFERENCES}

There are many published studies which bear in part on this topic.
The following ones will illustrate the methods of study and will indicate
the kind of conclusions which are generally drawn. This list is not
complete but is representative of the more recent ones written in English
and readily accessible to most students in the United States.

\noindent
\textbf{For Dairy Cattle:}

\begin{hangparas}{0.5in}{1}%
Aldrich, A. W., and Dana, J. W. 1917. The relation of the milk vein system to
production. Vermont Agr. Exp. Sta., Bul. 202.

Brody, S., and Ragsdale, A. C. 1935. Evaluating the efficiency of dairy cattle. Missouri
Agr. Exp. Sta., Bul. 351.

Copeland, Lynn. 1938. The old story of type and production. Jour. Dy. Sci. 21:295--
303.

Gaines, W. L. 1931. Size of cow and efficiency of milk production. Jour. Dy. Sci. 14:
14--25.

Garner, F. H. 1932. A study of some points of conformation and milk yield in Friesian
cows. Jour. Dy. Res., 4:1--10.

Gowen, John W. 1924. Milk secretion. pp. 32--49.

---. 1923. Conformation and milk yield in the light of the personal equation
of the cattle judge. Maine Agr. Exp. Sta., Bul. 314.

---. 1925. The size of the cow in relation to the size of her milk production.
Jour. Agr. Res., 30:865--69.

---. 1926. Genetics of breeding better dairy stock. Jour. Dy. Sci., 9:153--70.

---. 1931. Body pattern as related to mammary gland secretion. Proc. Nat.
Acad. of Sci., 17:518--23.

---. 1933. Conformation of the cow as related to milk secretion. Jour. Agr.
Sci., 23:485--518.

Swett, W. W.; Graves, R. R.; and Miller, F. W. 1928. Comparison of conformation,
anatomy, and skeletal structure of a highly specialized dairy cow and a highly
specialized beef cow. Jour. Agr. Res., 37:685--717.
\end{hangparas}

\noindent
\textbf{For Poultry:}

\begin{hangparas}{0.5in}{1}%
Bryant, R. L., and Stephenson, A. B. 1945. The relationship between egg production
and body type and weight in Single Comb White Leghorn hens. Va. Agr. Exp. Sta.,
Tech. Bul. 96.

Juli, M. A.; Quinn, J. P.; and Godfrey, A. B. 1933. Is there an egg-laying type of the
domestic fowl? Poul. Sci., 12: 153--62.

Knox, C. W., and Bittenbender, H. A. 1927. Correlation of physical measurements
with egg production in White Plymouth Rock hens. Iowa Agr. Exp. Sta.,
Res. Bul. 103.

Marble, D. R. 1932. The relationship of skull measurements to cycle and egg
production. Poul. Sci., 11:272--78.

Miller, M. Wayne, and Carver, J. S. 1934. The relationship of anatomical measurements
to egg production. Poul. Sci., 13:242--49.

Sherwood, Ross M., and Godbey, C. B. 1928. Construction of score card for judging
for egg production. Poul. Sci., 7:263--74.

Waters, Nelson F. 1927. The relationship between body measurements and egg production
in Single Comb White Leghorn fowls. Poul. Sci., 6:167--73.
\end{hangparas}

\noindent
\textbf{For Horses:}

\begin{hangparas}{0.5in}{1}%
Brandt, A. E. 1927. Relation between form and power in the horse. Trans. Amer.
Soc. Agr. Eng., 21:PM---3--4.

Dawson, W. M. 1934. The pulling ability of horses as shown by dynamometer tests
in Illinois. Proc. Amer. Soc. An. Prod. for 1933, pp. 117--21.

Dinsmore , Wayne. Learn to judge your horses and mules. Leaflet No. 196 of the
Horse Association of America.

Laughlin, H. H. 1927. The Thoroughbred horse. Carnegie Institution of Washington.
Yearbook No. 26, pp. 56--58. (See preceding and following yearbooks
for other short notes on this.)

Rhoad, A. O. 1929. Relation between conformation and pulling ability of draft
horses. Proc. Amer. Soc. An. Prod. for 1928, pp. 182--88.
\end{hangparas}

\noindent
\textbf{For Swine:}

\begin{hangparas}{0.5in}{1}%
Bull, S.; Olson, F. C.; Hunt, G. E.; and Carroll, W. E. 1935. Value of present-day
swine types in meeting changed consumer demand. Illinois Agr. Exp. Sta.,
Bul. 415.

Nordby, J.E. 1932. Type in market swine and its influence on quality of pork. Idaho
Agr. Exp. Sta., Bul. 190.

Scott, E. L. 1930. The influence of the growth and fattening processes on the quality
and quantity of meat yielded by swine. Purdue Agr. Exp. Sta., Bul. 340.
\end{hangparas}

\noindent
\textbf{For Beef Cattle:}

\begin{hangparas}{0.5in}{1}%
Black, W. H.; Knapp, Bradford, Jr.; and Cook, A. C. 1938. Correlation of body
measurements of slaughter steers with rate and efficiency of gain and with
certain carcass characteristics. Jour. Agr. Res. 56:465--72.

Hultz, F. S., and Wheeler, S. S. 1927. Type in two-year-old beef steers. Wyoming
Agr. Exp. Sta., Bul. 155.

Lush, Jay L. 1931. Predicting gains in feeder cattle and pigs. Jour. Agr. Res.,
42:853--81.

---. 1932. The relation of body shape of feeder steers to rate of gain, to
dressing per cent, and to value of dressed carcass. Texas Agr. Exp. Sta., Bul.
471.
\end{hangparas}

\noindent
\textbf{For Sheep:}

\begin{hangparas}{0.5in}{1}%
Bosman, V. 1933. Skin folds in the Merino sheep. S. African Jour. Sci., 30:360--65.

Cole, C. L. 1943. Type selection vs. record of performance in sheep breeding. Quart.
Bul. Michigan Agr. Exp. Sta. 26:31--33.

Hultz, Fred S. 1927. Wool studies with Rambouillet sheep. Wyoming Agr. Exp. Sta.,
Bul. 154.

---. 1935. A five-year study of Hampshire show sheep. Wyoming Agr. Exp.
Sta., Bul. 207.

Jones, J. M.; Dameron, W. H.; Davis, S. P.; Warwick, B. L.; and Patterson, R. E.
1944. lnfluence of age, type and fertility in Rambouillet ewes on fineness
of fiber, fleece weight, staple length, and body weight. Texas Agr. Exp. Sta.,
Bul. 657.

Joseph, W. E. 1931. Relation of size of grade fine-wool ewes to their production.
Montana Agr. Exp. Sta., Bul. 242.

Spencer, D. A., and Hardy, J. I. 1928. Factors that influence wool production with
range Rambouillet sheep. USDA, Tech. Bul. 85.
\end{hangparas}
\index{Type and production|)}