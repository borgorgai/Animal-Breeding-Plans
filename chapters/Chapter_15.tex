`\chapter{Aids to Selection --- Progeny Tests}
\label{cha:Lush_Chapter_15}
\index{Progeny test|(}

\begin{quote}
``The quality of a ram can usually be determined from his conformation
and from his get.'' ``You may judge them by their get if
their lambs are of good quality.'' --- Varro. The Husbandry of Livestock,
the first century \textsc{B.C.}
\end{quote}

\begin{quote}
``One may consider the ancestors as thoroughbred only when
all the progeny are thoroughbred.'' --- Thaer, in 1806
\end{quote}

By progeny test is meant estimating the individual's heredity by
studying its offspring. The general idea is an old one, as is indicated by
Varro's comments some 2,000 years ago. As long ago as 1826 Andre
recommended progeny testing as one of the main purposes for keeping
herdbooks for sheep. On page 299 in the USDA Yearbook for 1894,
the proving of bulls and the continued use of sires of proved excellence
are urged.

The principles of the progeny test come from the sampling nature of
inheritance. Each offspring receives from the parent a sample half of
that parent's inheritance. Each additional offspring receives another
independent sample from the same source. If one can find out what was
in several such samples, he will be fairly sure of what was in the parent.
A crude analogy may make the case clearer. If a barrel contained 100
apples, of which a certain number were ripe while the rest were green,
and if we could not get in the barrel to count the apples but could
count a sample half of them taken at random, we could tell within certain
limits what proportion of the apples in the barrel were ripe. There
would be a high degree of uncertainty about our estimate because this
one sample might by chance have contained considerably more or less
than a fair share of the ripe apples. If the sample half could be placed
back in the barrel and thoroughly mixed with the others and then a
fresh sample half could be counted and if that process were repeated
often enough, we could finally come to the point of knowing how many
apples were ripe and how many were green in the whole barrel,
although it would take several such samples for us to be sure that we
were within three or four of the right number.

The first practical difficulty encountered in using the progeny test is
that we do not know exactly what genes the offspring have. We can be
deceived by the effects of environment, dominance, and complex interactions
of genes in the offspring, just as we could in estimating the parent's
genes from its own appearance or performance. There is this important
difference, however: There are several of the offspring and the
deceiving effects of environment, dominance, and epistasis have an
opportunity to cancel each other. Thus, there is a chance for us to know
the average inheritance of the offspring with more certainty than we
know the inheritance of the parent. The parent is only one individual
and in it there has been no chance for the plus and minus errors to have
been canceled by averaging.

The second practical difficulty encountered in using the progeny
test is that each offspring also has received half of its inheritance from
its other parent. Since we usually do not know exactly what was in that
parent and will be still farther from knowing just what it contributed
to this particular offspring, we are often in doubt as to whether a certain
good quality in one offspring came from its sire or from its dam.
One way of overcoming this difficulty partially is to use all the available
information to estimate the other parent's inheritance; then allow for
that according to the general rule that the phenotype of an unselected
offspring will tend to average half way between the genetic values of its
two parents. Let \textit{X} represent the unknown genetic value of the parent
being progeny-tested and \textit{Y} be our estimate of the genetic value of the
other parent of an offspring which has a phenotype, \textit{P}.\index{Sire indexes} Then the most
probable value of \textit{P} is \(\dfrac{X + Y}{2}\) and the equation,
\(X = 2P - Y\) furnishes an estimate of \textit{X}, although not a very dependable one when
there is only one offspring, since Mendelian sampling variations, environmental
effects, etc., can have made \textit{P} in a particular case deviate
widely from the general rule. Also, of course, we may be rather wide of
the mark in our estimate of \textit{Y}. Another way of overcoming this difficulty
consists of progeny-testing an animal by breeding it to a large number
of different mates, in the hope that the merits and defects of those other
parents would just about cancel each other. Any general difference,
then, between the average of the progeny and the average of the breed
could be credited to the common parent. This method might, of course,
lead to large errors if the other parents were so selected that their average
merit was distinctly different from the breed average.

Theoretically, a third way of not being confused by what the other
parent transmitted is to use a special ``tester strain''\index{Lethal genes}\index{Tester strains|(} so constituted
genetically that their contribution to the inheritance of the offspring
will not hide the inheritance received from the parent being progeny-tested.
Occasionally this can be done in actual practice. Examples are
the use of red cows to test the \index{Homozygosis}homozygosis of bulls belonging to black
breeds, or the use of horned cows to test the homozygosis of polled
bulls.\footnote{If a black bull is mated to red cows and produces even one
red calf, he is known to be heterozygous. If he produces only black calves,
he is probably homozygous, the assurance of that increasing with the number
of calves thus produced. The laws of chance governing the case are such that
if a heterozygous black bull sires one calf from a red cow, that calf is just
as likely as not to be black; but, if he sires two calves, there is only 1
chance in 4 that both will be black; there is 1 chance in 8 that three
such calves will all be black, 1 chance in 16 with four calves, 1 chance in 32 with five
calves, etc., the chance being \(\left(\dfrac{1}{2}\right)^n\) where $n$ is the number of calves sired. Hence, if
a black bull has sired more than five or six calves from red cows and none of th~
calves are red, the chance that such a bull was really heterozygous is exceedingly
small.} This method has its principal use in overcoming the effects of
dominance but can also be used to prevent deceit by epistatic effects if
the genetics of the situation is known clearly enough that suitable tester
animals can be found or produced. Some of the most brilliant research
in genetics owes its success to the devising and use of such tester strains.
In economic animal breeding, however, this method cannot often be
used unless animals already produced for other purposes can be used
for testers. The production and maintenance of a suitable tester strain
would be expensive, if possible at all. The plan is not practical for
females because of the limited number of young they produce and
the economic impossibility of using them for several gestation periods
to produce to the tester males offspring which would then have to be
discarded because of the inheritance from their sire, no matter what
they proved about the genotype of their dams. The simplest form of the
plan could not be used to test for lethals since there are no tester animals
homozygous for those, and it would be difficult to maintain a
strain of farm animals heterozygous for a lethal gene. Generally the best
that can be done in such a case is to breed the suspected sire to a large
number of his own daughters although that requires more offspring
than if he could be mated to known \index{Heterozygosis}heterozygotes.\footnote{Berge, S.,
1934, \textit{Nordisk Jordbrugsforskning}, pp. 97--114} Even where tester
animals are available, as in the case of red and horned cattle, it would
not be often that the same animals could be used to test for more than a
few genes. A sire being progeny-tested for many genes would have to be
used on one set of tester animals to test him for some of his genes, on
another set to test him for a few more of his genes and on still other sets
to provide an adequate progeny test for many pairs of genes, simply
because a single strain of animals suitable for testing all genes would
not exist. From these and other considerations, it seems likely that the
progeny test in actual practice will nearly always have to be made incidentally
by studying the offspring produced by mating the sire to the
females to which he would be mated anyhow for other reasons.

It is believed by some, especially in the corn breeding work, that the
most accurate tests of the general combining power of different strains
can be made by testing them on poor or weak strains rather than on
medium or good ones. Theoretically, this seems likely to be true wherever
nearly all the favorable genes are completely dominant and the
best strains have in them nearly all of those desired genes; or wherever
being ``best'' is largely a matter of having a considerable margin of
unused merit, a factor of safety so to speak, which would not ordinarily
be needed. Crosses of the various strains to tester material known to be
weak in many ways might produce first crosses low enough in average
merit to show the differences in reserves or factors of safety among the
strains being tested. Such differences would not be detected if the
strains being compared were crossed on varieties which were themselves
good enough that nearly all the first crosses would be above the threshold
below which these extra margins of safety are needed. The practical
importance of using poorer-than-average stock for tester strains is uncertain.
It is not likely to be as useful with animals as with plants because
of the high cost of maintaining tester strains of animals. Attempts to
collect tester stocks by selecting poor individuals wherever they are
found would encounter practical difficulties in that the supposedly
poorer individuals would not be as poor genetically as they are phenotypically.
It would be difficult to discount for that correctly.
\index{Tester strains|)}

A third practical difficulty in using the progeny test is that the offspring
of a given individual are apt to have been born at somewhere
near the same date and to have been reared under much the same
environmental conditions. If there was anything unusual about that
environment and if proper allowance for that was not made, we will
credit or blame the heredity of the parent for something which was
really caused by the environment of the offspring. This is probably the
most important general limitation on the accuracy of the progeny test,
and there seems to be no automatic way of overcoming it. One can
merely study as closely as possible the environment under which those
offspring were tested and make such allowance as he thinks fairest for
any conditions which were not standard.

The principles involved in the progeny test of a sire are illustrated
in the equations below. In these equations, ``first error of appraisal''
includes all mistakes made in correcting the records of ``first dam'' and
of ``first offspring'' to standard environmental conditions and all mistakes
made in allowing for the effects of dominance and epistasis on
their records. These equations show how it is that the average error
from Mendelian sampling and the average error from mistakes of
appraisal become smaller as the number of offspring in the progeny
test increases. They also show why it is important that these individual
errors shall be as likely to be plus as they are to be minus. If that is the
case, the plus and the minus errors will tend to cancel each other so that
their average approaches zero as the number of offspring becomes large.
The errors from Mendelian sampling are certain to be thus unbiased,
but that may be far from true of the errors in appraising. Because the
offspring are apt all to have been reared under similar environment, any
peculiarity about that environment for which we have not corrected
perfectly is likely to have made most of the uncorrected environmental
errors plus or most of them minus. In such a case, the average environmental
error does not tend toward zero but toward a figure determined
by the kind and size of the systematic error made in standardizing the
records. Since such an error is doubled in getting the estimate of the
sire and does not tend toward zero as more offspring from the same herd
are tested, this is probably the major source of error in most progeny
tests-at least if there are enough offspring (more than four or five) to
make the errors from Mendelian sampling small. Errors in appraising
the effects of environment on the dams can also fail to be random, but
this is not so likely to be extreme as in the case of the offspring, since
the dams are less likely to have all been kept and tested under the same
environment, especially if lifetime records are used for them. Errors in
allowing properly for the effects of dominance and epistasis are apt to
be more important in the case of the dams than they are in the case of
the offspring because the offspring are usually unselected or nearly so,
while the dams are often somewhat selected. If dams were selected, considerable
regression toward the average of the breed should be allowed
in estimating their real breeding value from their records.

~\\
\setlength{\parskip}{6pt}
\noindent
\( \textrm{1st offspring} = \dfrac{\textrm{sire + first dam}}{2} \pm \textrm{1st Mendelian error} \pm \textrm{1st error of appraisal} \)

\noindent
\( \textrm{2nd offspring} = \dfrac{\text{sire + 2nd dam}}{2} \pm \textrm{2nd Mendelian error} \pm \textrm{2nd error of appraisal} \)

\noindent
\( \cdots\cdots\cdots\cdots\cdots\cdots\cdots\cdots\cdots\cdots\cdots\cdots\cdots\cdots\cdots\cdots\cdots\cdots\cdots\cdots\cdots\cdots\cdots\cdots\cdots\cdots \)

\noindent
\( \cdots\cdots\cdots\cdots\cdots\cdots\cdots\cdots\cdots\cdots\cdots\cdots\cdots\cdots\cdots\cdots\cdots\cdots\cdots\cdots\cdots\cdots\cdots\cdots\cdots\cdots \)

\noindent
\( \textrm{\textit{N}th offspring} = \dfrac{\textrm{sire + \textit{n}th dam}}{2} \pm \textrm{\textit{n}th Mendelian error} \pm \textrm{\textit{n}th error of appraisal} \)

\noindent
\( \textrm{Average offspring} = \dfrac{\text{sire}}{2} + \dfrac{\Sigma\textrm{dams}}{2n} \pm \dfrac{\Sigma\textrm{Mendelian errors}}{n} \pm \dfrac{\Sigma\textrm{errors of appraisal}}{n} \)

\noindent\label{eqn-page-198}
\( \textrm{Sire} = 2 \times (\textrm{average offspring}) -
	\textrm{average dam} \pm 2 \times (\textrm{average Mendelian error}) 
	\pm \allowbreak 2 \times (\textrm{average error of appraisal}) \)

\setlength{\parskip}{1pt}
~\\
It is important that the offspring be an unselected sample. Otherwise
the progeny test is biased by the selection practiced in choosing
which progeny to include. This bias is difficult to measure and discount.
To omit the poor offspring is unfair and misleading.

The result of the progeny test contains a term for the average merit
of the dams. If the dams were known to be typical of the breed, that
term will be the breed average and, since it will be the same for all
sires tested, will not affect the comparison of two sires. If the dams can
be assumed to have been a random sample of the breed, only a little
error is introduced by ignoring the dams in comparing two sires. If the
dams were not a typical or random sample of the breed, then neglect of
this term for the average merit of the dams can lead to serious error.

Increasing the number of offspring in the progeny test reduces the
errors of Mendelian sampling and the random errors in accordance
with the law of diminishing returns . That is, each additional offspring
reduces these errors but makes less reduction than the preceding one
did, so that further reductions in those errors are slight after the merits
of the first few offspring are known. Increasing the number of offspring
in the progeny test does not reduce the errors from systematic mistakes
in correcting for environment, dominance, or epistasis. If such systematic
errors are large in the population to be studied, little gain in
net accuracy is to be had by increasing the number of offspring much
past three or four.\footnote{Lush, Jay L., 1931, The number of daughters
necessary to prove a sire, \textit{Jour. of Dy. Sci.}, 14:209--20.}
Additional offspring do contribute some information,
but this is so little, compared with the large systematic errors
remaining, that efforts to increase still further the number of offspring
in the progeny test without first correcting for the systematic errors may
be described by the Biblical allusion about straining at a gnat and swallowing
a camel.

Requiring too many offspring for proving a sire can actually lower
the rate of progress by causing a smaller number of sires to be proved
and therefore preventing intense selection among them after they are
proved. As a numerical example, if 100 heifers born each year can be
used for proving sires, we can prove five sires per year if we require 20
daughters each, ten sires with IO daughters each or 20 sires with five
daughters each. If we must keep for future use the five best among the
sires proved each year, we would in the first case have to save them all,
regardless of their proof. Nothing would have been accomplished by the
proving. In the second case we would need to keep the half which had
the best proof. In the third case only the best fourth need be saved.
Progress would be fastest in the third case, although the ``proof'' would
be less accurate. On the average the extra errors in the proof would be
more than offset by the opportunity to cull more intensely. For maximum
progress the optimum number to prove each sire would be about
three daughters if selections were entirely on progeny test and if one
could neglect the extra costs of maintaining the larger number of bulls
while waiting for their proof. The existence of such costs makes the
practical optimum number higher, the amount of that shift depending
on the costs as well as on how extensively each bull is to be used after he
is proved. The essential point here is that it is no use to prove a sire
unless something is done on account of that proof --- something which
would not be done otherwise. Making the proof highly accurate can
actually retard progress if it is done at the expense of proving fewer
sires, as must be the case if the number of daughters required per sire is
increased in a cow population of fixed size.

\index{Polyallel crossing|(}
\index{Diallel crossing|(}\textit{Diallel crossing} is a method of progeny-testing introduced into genetic
literature by J. Schmidt\footnote{1919. Compt. Rend. Lab. Carlsberg, 14,
No. 6.} as a means to avoid setting any value on
the dams and to avoid correcting for the environmental circumstances.
The breeding values of two males are compared by breeding them at
different times to the same females and then comparing the average
merit of the two sets of progeny. By referring to the equations used on
page \textit{\textbf{198}} to illustrate the principles of the progeny test, it will be seen
that if the equation for one sire is subtracted from the equation for
another sire, the difference between the real breeding values of the sires
will be \textit{twice} the difference between their progeny averages plus or
minus terms for differences in the dams, the environments, and the
Mendelian sampling errors. Now if the dams are the very same individuals,
as they are in diallel crossing, the term for the difference
between dams will disappear. If the tests are carried out under the same
environmental conditions as, for example, when the two sires are used
contemporaneously, half of the females being bred to each sire the
nrst time and then each female being bred to the other sire the next
time, then the difference in the environmental conditions approache~
zero also. The error from Mendelian sampling can be made small by
having a large number of offspring. This leads to the very simple rule
that, if this plan can be followed, the difference in the breeding value of
two males is simply twice the difference in the average merit of their
offspring. The plan has practical difficulties in such cases as dairy cattle,
where one can measure production only in daughters and cannot be
sure of getting from each cow a daughter by each sire, but it seems to
have advantages enough to deserve wider use than it has yet received.
It can be extended with more difficulty to the simultaneous comparison
of more than two sires. That is being done experimentally with swine
\footnote{Kudrjawzew, P. N., 1934, Polyallele Kreuzung als Pr\"ufungsmethode
f\"ur die Leistungsf\"ahigkeit von Zuchtebern. Z\"uchtungskunde, 9 (No. 12):444--52.}
but how it will work in practice is not yet certain. The case of four
boars requires eight groups of sows which are indicated by letters A to
H in the following mating plan:

\begin{table}[h]
	\centering
	\caption{\textsc{Mating Plan in Polyallel Crossing}}
	\begin{tabular}{l|c|c|c|c}
		\hline
		\hline
		 				& \multicolumn{4}{c}{Boar Number} \\
		 				\cline{2-5}
		 				& 1 &	2 &	3	& 4	\\
		\hline
		First season	& A and B	& C and D	& E and F	& G and H \\
		Second season	& H and C	& B and E	& D and G	& F and A \\
		\hline
	\end{tabular}
\end{table}

\noindent
The diallel comparison of boars 1 and 4 is provided by the progeny of
sows in groups A and H. The comparison between boars 1 and 2 is provided
by the progeny of the sows in groups B and C, etc. There is no
direct comparison between boars l and 3, but each of these boars is compared
directly with boar 2 and also with boar 4. Two indirect
comparisons of l and 3 are possible from that. The ``ring'' arrangement
provides that, if for any reason the comparison between two boars is not
completed, there is still a ``chain'' arrangement by which any boar may
be compared with any other unless there is a second failure in some
other comparison.
\index{Diallel crossing|)}
\index{Polyallel crossing|)}

In an unselected population where individual merit is equally well
known for all animals, one offspring is as reliable as a parent in indicating
what an animal's inheritance is. But an individual can have only
two parents, while there is no such biological limit on the number of
offspring it can produce! Because of that, the progeny test can surpass
the pedigree as an accurate guide to an individual's inheritance, provided
the offspring are half-sibs, and can equal it if they are all full sibs.
In an unselected population with the outward merit of each animal
equally well known and no environmental correlations between relatives,
a progeny test based on three offspring where the other parents
can be assumed to have been a random sample of their breed, is equal in
accuracy to all that could be learned by studying the animal's pedigree\index{Pedigrees as aids to selection}.
Two offspring are enough to make the progeny test attain the same level
of accuracy if the merits of the other parents are known and discounted.
These simple conditions never prevail exactly. Usually the individual
merits of the offspring are not as certainly known as the individual merits
of the ancestors because the latter are older and have had a longer
time in which to manifest their qualities. The offspring are more apt to
have been reared and tested under the same environmental conditions
which, not being perfectly discounted, are apt to make the progeny test
less reliable than under the simple conditions. A circumstance which
operates in the opposite direction is that the ancestors were usually
selected to some extent, whereas the offspring usually are almost or
quite unselected. An extreme example of this effect of selection among
the ancestors concerns the occurrence of red in black breeds of cattle.
Since all red animals are culled from the pure breed, the pedigree would
not be of any help in locating the heterozygous animals (unless perhaps
it were one of those few pedigrees which contained information about
some red collateral relatives), but the production of even one red offspring
would be positive proof of heterozygosis.

These usual exceptions to the simplest theoretical conditions leave
the question of when progeny test generally becomes more accurate
than pedigree somewhat uncertain in actual practice. Perhaps it is just
as well to think of pedigree and progeny test as about equal when there
are two or three offspring, although that does not do justice to the usefulness
of the progeny test where the ancestors have been highly selected.
On the other hand, such a rule does not do justice to the usefulness
of pedigree where the offspring have all been raised under some environmental
peculiarities about which we do not know or for which our
corrections have not been entirely unbiased.

The point at which the progeny test becomes more accurate than
the individual merit of the animal itself is of special interest. A numerical
solution has been given\footnote{Lush, Jay L., 1915, Progeny test and individual
performance as indicators of an animal's breeding value, \textit{Jour. of Dy. Sci.},
18:1--19.} for the case of a hitherto unselected population
where the individual merit of the parent and of each of its
offspring are equally well known. In that case at least five offspring are
necessary £or slightly hereditary characteristics where there is no correlation
between the offspring for any other reason than that they are
related through this one parent. If the characteristic is highly hereditary,
more than five will be necessary. If there is a correlation between
the offspring for other reasons, more offspring will be necessary; and if
that correlation is as high as $+$ .25, and if correction for it is not made,
it will be impossible for the progeny test to become as accurate an indicator
of the parent's heredity as the parent's own merit is. In actual
practice the parent's individual merit will usually be known better than
that of the offspring; but on the other hand, the parents will usually be
to some extent the survivors of previous cullings for individual merit
and for pedigree. I£ there has been much selection of that kind, the
practical situation is that the possibilities for culling by pedigree or by
individuality have been partially exhausted before the progeny test
becomes available. The progeny test is thus a fresh opportunity to cull
from an entirely new direction. Hence, in a population of selected parents,
the progeny test is relatively more useful than the above figures
indicate.

Progeny tests are most useful for traits which can be expressed
in only one sex (e.g., milk production, egg production, prolificacy, etc.).
In such cases study of the individual animal of the sex which cannot
express the trait does little if any good. Selection in that sex is limited
to the basis of pedigree and progeny tests, although individual merit is
also available as a basis for selection in the other sex.

The fundamental genetic effect of progeny tests is that they permit
selections to be more accurate and hence more effective because they
prevent the breeder from being deceived as much by the effects of environment,
dominance, and epistasis as he might othewise [\textit{sic}] be. They do
not alter any genetic process.

All the initial selections must be made while the animals are still
young and many of the final selections before they are old enough to
have any progeny of known merit. Therefore, the progeny test is
applied only to animals which have already met minimum standards of
pedigree and individuality. Some loss is incurred by culling without a
progeny test some which would have proven to be better genetically
than their pedigree or individuality indicated, but it is utterly impossible
to test the progeny of them all. The ideal is to select on pedigree
and individuality to some extent but not so much that all one's freedom
to select will have been exhausted before any of the offspring can be
examined. Within limits dictated largely by costs and convenience but
partly by the accuracy of selecting on different bases, the breeder should
strive to sample enough of those with good pedigrees and individuality
that he can still do considerable culling when the results become
apparent.

Breeders have always made general but somewhat unsystematk
attempts to use the progeny test. They have done this by seeking sires
and dams or the sons of sires and dams which had produced unusually
good offspring. ``Get of sire'' and ``produce of dam'' classes have been
included in shows in nearly all countries for many years. In recent
years a pronounced interest in progeny testing has developed, especially
in dairy cattle and in poultry. In meat animals the widest systematic
use of the progeny test has been the Danish system of testing litters of
swine\footnote{Lush, Jay L., 1916, Genetic aspects of the Danish system
of progeny-testing swine, Iowa Agr. Exp. Sta., Res. Bul. 204} which has
also been adapted and used widely in Sweden, Canada, The Netherlands, and Germany.

In a certain sense all sires and many dams become progeny-tested
eventually, but usually they are dead by that time\footnote{In a survey of the
ages of 35,000 dairy bulls in Michigan, it was found that 94 per cent were
slaughtered before they reached three years, although a bull cannot be
proved before he is five years old. Michigan Quart. Bul. 15 (No. ll):143. See also
Iowa Sta. Bul. 290, \textit{The ages of breeding cattle and the possibilities of using proven
sires}.} and the information can be used only in pedigree estimates.

Where there is sex-linkage\index{Sex linkage}, offspring show more about the inheritance
of the opposite-sexed parent than they do about the inheritance of
the parent of their own sex. This is not often important, but some
allowance of that kind can be made where sex-linkage is suspected.

\section*{SUMMARY}

\begin{enumerate}
\item ``Progeny test'' is a general term for estimating the breeding value
of an animal by studying the characteristics of its offspring.

\item The things which may keep the progeny test from being perfectly
accurate are: (1) The sampling nature of inheritance makes it possible
for the parent to transmit to its offspring inheritance which is better or
is worse than is typical of it; (2) the offspring receives half of its inheritance
from its other parent, and the breeding value of that other parent
may be distinctly different from the average of the breed; (3) environmental
effects, dominance deviations, and epistatic deviations may
deceive us in our estimate of the real merit of the offspring or of the
other parent.

\item Errors coming into the progeny test because of the sampling
nature of inheritance may be made as small as we please by getting a
sufficiently large number of offspring. Where there are five or more offspring
the errors from this source are usually small in comparison with
errors from the other sources.

\item Where the other parents can be considered a random sample of
the breed or are known to be typical of the breed, errors in the progeny
test from neglecting the merit of those other parents are zero or
approach zero rapidly as the number of offspring increases. Where the
breeding merit of the other parents is distinctly above or below the
breed average, allowance for that can best be made by estimating the
breeding value of the parent being tested as equal to twice the average
merit of its offspring minus the average merit of the other parents of
those offspring, with some allowance for regression toward the breed
average or herd average.

\item Errors coming into the progeny test through not making fair
allowance for environmental conditions which were not standard for
the offspring are usually the most serious limitation on the accuracy of
the progeny test. Random errors of this kind are reduced rapidly by
increasing the number of offspring and thus allowing the plus and the
minus errors to cancel each other. Systematic errors are not thus reduced
and are likely to be important. The only remedy for this is the general
one of studying closely the conditions under which the offspring were
reared and tested and making the best allowance one can for those.
When errors of this kind are important, not much information is
gained by increasing the number of offspring past three or four.

\item The progeny test can become more accurate than a pedigree estimate
in a population as a whole, when there are more than three offspring,
but this depends on whether the individual merits of the offspring
are as certainly known as the individual merits of the ancestors,
on how much environment the offspring have had in common and on
how much the variation among the ancestors has already been reduced
by selection among them.

\item In a hitherto unselected population, individual merit is usually
more dependable than progeny test unless: (1) there are at least 5 offspring,
(2) there is no environmental correlation between the offspring,
(3) the characteristic is not very highly hereditary, and (4) the individual
merits of the offspring are known at least as accurately as the merit of
the parent being tested. In actual practice, (2) and (4) are not usually
fulfilled, but they may be more than offset by the fact that selection on
the basis of pedigree and individuality may already have exhausted
much of the possibilities in such selection, while the progeny test comes
as a fresh source of evidence from a new direction.

\item Progeny tests are most useful for characteristics which only one
sex can express or which, like many carcass characteristics, cannot be
measured until the animal is dead.

\item Progeny tests are next most needed for characteristics which are
only slightly hereditary and for which individual selection is therefore
not very accurate.

\item The chief practical limitation on progeny tests is that they come
so late in the animal's life that most of the decisions about culling or
using an animal for breeding must already have been made. Therefore,
progeny tests have their widest use in making pedigree selections more
accurate by pointing to the sires and dams whose offspring are most
likely to be worth saving for breeding.
\end{enumerate}
\index{Progeny test|)}