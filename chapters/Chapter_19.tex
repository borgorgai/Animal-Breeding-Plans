\chapter{The Show Ring and Animal Breeding}
\label{cha:Lush_Chapter_19}
\index{Shows and fairs|(}

The original purpose of fairs was to provide a place for trading. In
some parts of Europe, even yet, the man who has a few animals for sale
may take them to a weekly fair. If no buyer makes an acceptable bid, he
takes them home again to wait for the next fair. It is a small step from
exposing animals for sale to exposing them for having their merits
appraised by other breeders. The exhibition of animals not intended
for sale was a prominent part of animal breeding practices even as far
back as the early history of the Shorthorn breed. Charles Colling's
famous ``Durham Ox'' was started in 1801 on a tour of exhibition which
lasted for six years. This tour was more like the sideshows than like the
show rings of today. The Colling-bred ``White Heifer That Traveled''
started on a similar tour a little later. The Booths were famous for showing
their breeding stock at many fairs. Thomas Bates on occasion exhibited
his Shorthorns and prided himself on his ability to judge, although
he was outspoken in his criticisms of the evils of overfitting and of keeping
cattle just for showing. According to Wriedt, the first public show
corresponding to our modern livestock shows was held in 1798 in Sussex,
England. The first public show in Denmark for all kinds of livestock
was held in 1810. The first public show in Wurtemberg was in
1817. Most of this early showing was for advertising purposes, and the
premiums offered were small or consisted only of trophies. According to
Juli the first poultry show in the United States was held in Boston in
1849. Poultry showing in England began at about the same time.

\section*{THE SHOW RING IN RELATION TO BREED IMPROVEMENT}

There are two ways in which the show ring may affect breed
improvement. First, it may keep the breeders informed about the ideals
of the breed. If they use that information in their own selections, the
show ring can be an important factor in guiding the direction in which
the breed is to be changed. Secondly, the show ring might be used to
find the best animals in the breed to such an extent that breeders could
accept the show ring placings as guides in buying and selling their
breeding animals. While this condition, of course, is never entirely
reached in any land, yet the advertising and popularity which certain
animals may acquire through their high winnings in the show ring may
go far to get them or their sons or daughters used extensively in many
herds which otherwise would never have sought them. This may have
some effect upon the genetic composition of the breed if it is done year
after year, since it constitutes a mild grading-up process toward the
prize winners and the herds from which they came. This is a form of
selection favoring the genes which are most frequent in the type of animals
which the judges place highest. Even though the judge sees but a
tiny fraction of the animals of the breed, his approval or disapproval
may help determine which animals become paternal grandsires or great
grandsires of the breed. Since one herd rarely is a prominent prize winner
for more than a few years, this does not often emphasize any one
animal or family in a way which could be called linebreeding. It is
emphasis of an ideal, rather than emphasis of an animal.

In the shows in the United States and Britain, the main object is to
emphasize the visible ideal which is held by the breeders of each breed.
No attempt is made in American and British shows to judge pedigrees.
The few attempts to give some weight to production records, in addition
to what is visible to the judge, have met with only partial approval
from those who tried the experiment. The shows, especially the larger
ones, give tremendous emphasis and advertising value to championships
and first places, out of all proportion to the usually small differences
in real merit between first and second or third place animals. The
main thing is to exalt before the public the most nearly ideal combination
of visible characteristics which can be found and to give the breeders
a clearer picture of the perfect animal to guide them in their own
selections. It is only incidentally that attention is directed to the animal
for its own sake. The major fairs in the United States usually perform
well this function of e?'alting the ideal. The larger fairs do not give
much help to the beginning breeder , since the placings he sees mostly
concern small differences between animals nearly all of which are good
ones, and the interest in judging is mostly centered on the placings of
the top two or three animals.

In the 4-H Club shows in the United States and in many of the
shows in continental European countries, there have been earnest
attempts to remedy this and to make the shows a place of instruction in
judging from top to bottom. This is intended to benefit the large part
of the public and the considerable fraction of breeders who have not
become experts in judging and perhaps never will, but who wish some
information or training in the kind of judging they themselves can do
in their own herds. In their applications of judging they may have an
opportunity to cull perhaps the poorest third or fourth of their females;
or perhaps they must select a sire from among a group of moderately
good males, no one of which is really of championship caliber. The
devices used to make these shows more instructive to the beginner and
to the general public mostly concern: placing all animals shown, no
matter how poor they are; having grades or descriptive terms which are
kept as nearly constant as possible from show to show and from year to
year so that the written record of the animal's show ring placings may
have a standard meaning; stabling or tying the animals during the fair
in the classes and in the order in which they were placed so that visitors
may see and study the placing at any time during the fair; minimizing
showmanship in the placings; and having as many breeders participate
in the judging as is reasonably possible. It is probable that some of these
practices can with advantage be adapted to American conditions for
tho se shows where a majority of the animals come from nearby farms
and where most of those who show are relatively inexperienced.

The show ring can do only a small amount toward ranking the animals
of the breed in the order of their real breeding value. In the first
place, only a tiny portion of the total number of animals of the breed
are shown. Table~\ref{tbl:Lush_Table_17}
shows this in a general way, using Iowa as an example
because its state fair is large and there are many purebred animals
in Iowa. The figures in this table give only a rough idea of the extent of
actual participation in showing, however. Besides the qualifications
listed under the table and the fact that the years are not identical, many
of the exhibitors and animals were from outside the state.\footnote{In
Denmark, where participation in the fairs is more general and each owner
can show only in the district where he lives, about 6,000 bulls and 10,000 
heifers and cows altogether are shown annually. This is about 7 or 8 per
cent of the bulls but only about 0.4 per cent of the females among all
the cattle of Denmark over one year old.} In the second
place, differences in showmanship, grooming, and such things may
affect the animal's show ring ranking, although they have no bearing on
its breeding worth. Such practices may even prevent the judge from
ranking the animals as nearly in the order of their breeding merit as he
could if no preparation for showing were made. An example is length
of wool, which is an important part of the practical merit of a sheep,
especially of sheep of the Merino and Rambouillet breeds. Yet the
length of fleece which the sheep wears when it enters the show ring
may be so altered by shearing early, stubble-shearing, blocking and
trimming, etc., that the judge cannot afford to pay as much attention to
it in the show ring as he usually would if he were culling his own sheep
where he knew that all had had substantially the same length of time
in which to grow the fleeces he saw before him. In the third place, the
temporary condition of the animal must count for much because of the
judge's duty to set before the public an animal which at that very
moment comes nearer the visible ideal of the breed than any other animal
in the ring. In the fourth place, the judge's ability to find the best
breeding animals, even if these other obstacles could be overcome, is
limited, of course, to the correlation which exists between outward
appearance and real breeding value. In the fifth place, it is difficult to
compare a placing in one show with a placing in another unless one was
present at both shows and remembers what kind of animals were present
at each. What comparison could one make from the information
that one sire won first as an aged bull at the Page County Fair in 1935, a
second sire was third as a two-year-old at the Iowa State Fair in 1933,
while a third sire stood tenth as a Junior Yearling at the National Dairy
Show in 1931?

\begin{table}
	{\centering
	\caption{\textsc{Numbers of Registered Animals in Iowa in the 1930 Census and Numbers
Exhibited at Various Iowa Fairs From 1930 to 1936}}
	\label{tbl:Lush_Table_17}
	\begin{tabular}{L{2cm}|C{1.5cm}|C{1.5cm}|C{1.5cm}|C{1.5cm}|C{1.5cm}}
		\hline
		\hline
%						& \multicolumn{2}{>{\hsize=\dimexpr2\hsize+2\tabcolsep+\arrayrulewidth\relax}X}{Farmers or Breeders Concerned}	& \multicolumn{3}{>{\hsize=\dimexpr3\hsize+2\tabcolsep+\arrayrulewidth\relax}X}{Number of Animals} \\
						& \multicolumn{2}{|C{3cm}|}{Farms or Breeders Concerned} & \multicolumn{3}{|C{4.5cm}}{Number of Animals}\\
		\cline{2-6}
		Kind of Animal	& Farms With Registered Females in 1930	& Exhibitors$^{*}$ at State Fair in 1936	& Registered Females on Farms in 1930	& Registered Animals Exhibited at State Fair in 1936	& Average Number Exhibited at County and District Fairs Each Year, 1930 to 1935$^{\dagger}$ \\
		\hline
		Draft horses		& 1,974					& 72		& 5,241					& 383	& 4,355$^{\ddagger}$ \\
		Beef cattle			& 5,556					& 113		& 60,685				& 713	& \\
		Dairy cattle		& 5,297					& 112		& 33,272 				& 719	& 12,809$^{\S}$ \\
		Dual-purpose cattle	& 437 					& 19		& 4,003					& 200	& \\
		Sheep				& 1,352$^{\parallel}$	& 73		& 15,677				& 830	& 4,551 \\
		Swine				& 7,446$^{\parallel}$	& 226		& 66,189$^{\parallel}$	& 2,238	& 15,118 \\
		\hline
	\end{tabular}
	}
	$^{*}$Including 4-H Club members who exhibited registered animals.\\
	$^{\dagger}$The number of these fairs ranged from 76 to 82, averaging 78 in the six
	years. The number exhibited each year is the sum of the numbers exhibited at each
	of these fairs. Each fair was about two to four days in length, and they were
	scattered over a period of about nine weeks during August and September. Doubtless
	many animals, were exhibited at more than one of these fairs and hence are counted
	more than once in these numbers, but no way was found to estimate how many of these
	duplications there were.\\
	$^{\ddagger}$Includes all horses without distinction between light and draft horses.\\
	$^{\S}$Includes all cattle.\\
	$^{\parallel}$All registered animals, whether male or female.
\end{table}

Probably the ideals of the show ring are usually those of a majority
of the breeders, but it is not certain whether that is because the show
ring leads the breeders or whether it merely reflects their current opinions
after those have been formed by other circumstances, just as the
driftwood on a river shows the course and speed with which the water
moves but does not cause or guide that movement. Changes in ideals
do sometimes occur; sometimes those get well started even against the
disapproval of the current show ring ideals. A case of that was the
marked change in ideals in the Poland-China and most other American
breeds of swine between 1910 and 1920. When such changes are in process,
the show ring may help them to spread more rapidly by giving the
breeders occasion to meet and discuss the subject with examples before
them.

In spite of its imperfections, no good substitute has yet been devised
for the show ring as a means of indicating what kinds of individuals are
best in the breeds of beef cattle, hogs, sheep, and draft horses. Even in
animals such as dairy cattle and poultry, where there are reasonably
simple and accurate tests for production, the show ring has not been
displaced by these tests. Among all farm livestock, the Thoroughbred
and Standardbred horses come nearest to relying upon production records
with little use for shows. Among poultry breeders there is a rather
wide gap between those who breed for production and those who breed
for show. In dairy cattle there is a similar but less extreme divergence
of opinion as to the usefulness of the show ring.

\section*{SPECIAL FEATURES OF SHOWS IN CONTINENTAL EUROPEAN COUNTRIES}

Brief mention will be made here of some show ring practices in non English-
speaking countries. Some of these might be useful, especially
in local fairs, if adapted to American conditions, or may be interesting
because of their distinct contrast with the practices common here.

All animals exhibited in a class are usually placed from top to bottom,
although it is permissible for the judges to indicate that two or
more are equal. The animals are usually divided into at least three and
not more than five classes. Prize money and usually the permanent records
make no distinction between animals in the same class. Thus, in a
single class of bulls there may be four ``first prize'' bulls, six ``second
prize'' bulls, five ``third prize'' bulls, six ``fourth prize'' bulls, and three
judged too poor to receive any prize. Those who are at the fair will
know the individual ratings within each class. Those individual ratings
are printed in the list of awards prepared immediately after the judging
is completed; but they may not go into the permanent records and
will not appear in the pedigrees of descendants of these animals. In
those pedigrees it will be stated simply that this animal was ``second
prize'' at a certain fair in a certain year. Every reasonable attempt is
made to keep the standards of judging constant so that ``second prize''
will mean the same thing at all fairs in all years in any one country.

In the case of cattle and horses, classes which are judged together
are tied together during the daytime in the order of their placing until
the fair is over, so that a visitor can study at almost any time during the
show the placing in any class in which he is interested. He does not need
to be on hand to see it judged. Physical difficulties may prevent that
with swine and sheep. Over each animal or pen is posted prominently
its classification and often its score, perhaps accompanied by the judges'
criticisms and commendations of certain things about it. The catalogues
contain for each animal the production records and scores or
classifications of its ancestors.

Sometimes pedigrees are classified or scored also. Practice varies
about whether (1) the pedigree score and the individuality score are
combined into a single net score for the animal, (2) separate prizes are
given for pedigree classification and for individuality classification, or
(3) the prizes are given only for individuality, while the pedigree scores
are printed merely for information. Since 1930 at the German national
show cows must have records of production to be eligible for showing.

Many breeders participate in the judging. Sometimes they work
singly and sometimes in committees. Where the classes are large, there
may be almost a separate committee for every class.

Much use is made of progeny groups of one kind or another. Those
vary more in the rules as to numbers required, etc., than do the ``get-of sire''
and ``produce-of-dam'' classes in American shows. At some of the
Danish shows at least two-thirds of the progeny of an older bull must be
exhibited if he is to gain a prize for type. Some of those progeny groups
may be judged on farms before the show. That is often done with
stallions.

Showmanship is minimized in many ways. Usually the attendant
makes little effort to pose his animal. In the bull shows in Switzerland
the judging is done behind closed gates and not even the owner is
allowed to be present. When the judges finally get a class placed in
what they believe is the correct order, the bulls are allowed to stand
for a while tied to the rail. Then the judges come back to look for
defects which may become evident after the animals have stood for a
time, and to make sure that the placing is satisfactory. The judging
takes place in the first half day in these Swiss shows, but the animals
must remain on exhibition during the daytime for three or four days.

In many countries there is more selling of exhibited animals than is
usual in the United States . More than half of the bulls at these Swiss
shows may change owners before the show is ended. In the Argentine
shows nearly all prize-winning animals are auctioned afterward. The
owner may withhold his animal from the sale if he wishes, but this is
not often done. In case he does that, the owner pays the management a
fee to cover expenses and to pay the auctioneer what he would have
received if this animal had been sold.

\section*{BUSINESS ASPECTS OF THE SHOW RING}

The show ring is one of the best channels for advertising surplus
stock. Many potential customers will not inquire whether the animals
which won the prizes are closely related to those which the breeder is
offering to sell them. Because of this, breeders who have many animals
for sale can sometimes pay large prices for good show animals owned by
some one who is not intending to exhibit them. The money thus spent
is an advertising expense, just as surely as if it had been spent for newspaper
space. It may be profitable if it helps keep the name of the
breeder before his potential customers in a favorable light and if he has
many animals to sell.

The show is an excellent place to meet other breeders and exchange
ideas and experiences which may be of considerable practical value. In
this way one can do much to keep informed on matters of concern to
the breed and can learn of events or changes of reputation which he
would not otherwise learn so soon. Sometimes the members of the
breed association present will hold a meeting some evening after the
judging is over to discuss matters which can be handled only in a co-operative
manner.

The show ring provides an opportunity to learn judging, at least
among the better animals, or to keep up to date the judging knowledge
one already has. One will learn much by standing at the ringside, making
his placing before the judge does, and then trying to see why the
judge's placing was different from his own.

The practice of fitting out a show herd and going from one show to
another on a long circuit insures that the average show ring merit of the
individuals exhibited at each fair shall be higher than if there were no
circuit system. Thus the show ring will more nearly achieve its object of
showing the public the ideal which is held by each breed. Against this
must be balanced the fact that it tends to destroy local participation in
the fair. Many a breeder, who might take his animals to the nearest fair
if he did not have to compete with professional showmen, will leave his
animals at home and go to the fair as a ringside spectator under the
present system. The long circuits promote professionalism in showing
because the skill of the showman has a chance to be rewarded in many
shows, not merely in one. The high rewards for success in professional
showmanship are an incentive toward such practices as surgical operations
to correct defects and toward all manner of deception in showing
the animal. The show tends to become more of a contest between showmen
and less of a court of inquiry as to which animal really would
be most useful for further improving the breed. In doing so, it may
acquire some of the sporting interest which attaches to a horse race, but
it becomes less useful to agriculture. The long show circuit may keep
the herd away from home for a long time when the animals should be
used for breeding.

Professional showmen sometimes take advantage of the circuit system
by exhibiting several different breeds of the same species of livestock,
especially at fairs where the competition will be light. This is
particularly common with exhibitors of sheep where prize money is
offered for so many different breeds. Such show herds or flocks are sometimes
called ``carnival outfits'' or ``gypsy herds.'' Few men are successful
breeders of more than one breed of each kind of livestock. The most
which can be said in economic justification of the carnival outfit is
that it will advertise a breed in a region where that breed is little
known. This may lead to some increase in sales by breeders of that
breed and at rare intervals may be the means of introducing to a region
a breed which has some real usefulness there but would not otherwise
find a foothold so soon. The managers and directors of fairs are
usually reluctant to reduce the prize money offered for the rare breeds
to quite as low a level as would be proportional to the number of them
which are bred in that region.

Many animals which played a prominent part in breed history were
themselves prize winners, but there have also been some which were not
shown or did not place high and yet did have more influence on the
breed than any of their contemporary prize winners. Champion of England,
who, more than any other one animal, was responsible for the
``Scotch type'' of Shorthorn, was of doubtful individual merit as a calf
and was nearly discarded without being tried as a sire. In the Hereford
breed Anxiety 4th was not shown, although it is said that he was an
excellent individual. His owners regarded him as too valuable to risk
fitting for showing. His sire, Anxiety, had been lost in just that way.
Many of the sons and grandsons of Anxiety 4th were shown, but Beau
Brummel, the grandson who became the most influential animal of the
whole breed, was shown only once and placed fifth in his class that time.
It is related that he would have ranked higher if he had been especially
fitted for showing. The noted Shorthorn sire, Avondale, was fourth in
his class at the 1908 International but later, as a sire, excelled the three
which placed above him. It does not seem that any valid general conclusion
can be drawn from such individual cases. They demonstrate
that show ring ranking is not an infallible guide to future success as a
breeding animal, but not even the most enthusiastic admirer of the
show ring would maintain that it is.

\index{Epistatic effects|(}
Whether an animal has an important influence on a breed depends
on the opportunity it has and on chance circumstances, as well as on
the kind of heredity it really has. As a rule those animals which stand
high in the show ring are given a better opportunity as breeding animals
than those which do not stand so high, but there is much variation
in this. The financial circumstances of their owners and other incidental
circumstances, which have no relation to an animal's breeding worth,
are often the controlling factors in determining what influence an animal
shall have on the breed. The Hereford bull, Beau Brummel, and
the Aberdeen-Angus bulls, Black Woodlawn and Earl Marshall, were
offered for sale to grade herds while yet young; but, fortunately for
their breeds, the sales were not completed and the bulls were used for
many years in purebred herds. The Percheron stallion, Brilliant 1899,
was used for nearly a dozen years in France and then was sold to America,
where he was used but one year on purebred mares.\footnote{See pp.
237--39 in \textit{A History of the Percheron Horse}, by A. H. Sanders and
Wayne Dinsmore, Chicago, 1917.} After that he
was used for nearly 15 years on grade mares only. The colts he sired in
France and the kind of grade colts he sired in the latter half of his life
indicate that the history of the Percheron breed might have been materially
different if he had stood at the head of a stud of purebred mares
during the latter half of his life. It is said\footnote{Page 6 of
\textit{The Breeders Gazette} for February, 1932.} that Mr. Gentry once offered
to take \$25 for Longfellow, the Berkshire boar who afterward became
the most famous sire of his breed. In fact, the buyer, who was merely
looking for a boar to ship to a tenant, was given his choice of two pigs
at that price. Mr. Gentry, trying to help him out, suggested Longfellow
and the suspicious buyer promptly chose the other pig!

Since differences in visible merit are partly determined by the genes
the animals have, it is to be expected that, if all animals were given
equal opportunity, prize winners would usually have a higher proportion
of prize-winning offspring than would breeding animals which
were not prize winners themselves. But the animals' opportunities to
be shown and to be used for breeding are different and little is known
definitely about the heritability of differences in show ring merit. Rice
found .21 for the regression of daughter on dam in the official type
classification of Holstein-Friesians but herd differences in environment
might have been responsible for more of 'this than heritability was.
Proportion, symmetry, and balance are emphasized so much in the
show ring that epistatic gene interactions seem likely to be important
causes of differences in show ring placings. For these reasons it is impossible
to say how many more of the offspring of prize winners will be
capable of winning prizes themselves than will be the case among the
offspring of those which did not win prizes. There have been many
studies of the pedigrees of groups of prize winners, but only a few of
these have included comparisons with the pedigrees of a representative
sample of the whole breed. Those few have indicated that the pedigrees
of the prize winners are substantially the same as the average
pedigrees of the breed, so far as concerns ancestors much more than
two or three generations back in the pedigrees.
\footnote{\textit{Jour. of Heredity}, 22:245--49, and 27:61--72.} Some of the animals
prominent as parents, grandparents, or great grandparents of the prize
winners have not been so prominent in the average pedigrees. It seems
possible to interpret this either as meaning that the prize winners come
largely from only a few contemporary families to which the breed will
later be graded up, or as resulting incidentally from the fact that only a
few owners make a regular practice of showing at the large fairs, and
any sires used extensively in their herds will almost inevitably be prominent
in show ring winnings a few years later. In either event, it can
hardly be maintained that the prize winners constitute very distinct
families or strains within the breed.
\index{Epistatic effects|)}

\section*{SUMMARY}

The show ring is a means of emphasizing the current ideals of the
breed. If breeders are guided much in making their own selections by
noting the types of animals which are placed high in the show ring, the
show ring can have an important part in guiding the direction in which
the breed average shall move.

It is not certain whether the show ring really causes the changes in
the ideals of the breed or merely reflects the ideals currently held by
a large portion of the breeders. There have been times when the breed
ideal changed, even against opposition from the show ring. Probably
the show ring cannot lead the whole breed far in a direction contrary to
the ideals of the commercial breeder.

To a limited extent the show ring may help in rating the animals
of the breed according to their breeding value. It is not very effective in
this because: (I) the correlation between outward appearance and real
productiveness is low for many characteristics; (2) so few of all purebred
animals are shown; (3) considerable attention is paid to fitting, to
temporary conditions, and to showmanship; and (4) many important
things about which the breeder may know, such as amount of milk and
fat produced by dairy cattle, number of pigs weaned by sows, length of
fleece on sheep, etc., must for practical reasons be given only a little
attention by the judge since he cannot know exactly what those were.

In spite of these limitations there is not yet any good substitute for
the show ring in measuring the general merit of the meat animals . Even
breeders of animals which, like dairy cattle, have reasonably complete
measures of productiveness not connected with the show ring continue
to make extensive use of shows.

In a business way the show ring is an important means of advertising.
The fairs offer opportunities to make sales and to exchange news
and ideas with other breeders.

\section*{REFERENCES}

\begin{hangparas}{0.5in}{1}%
Eckles, C. H. 1933. We can improve our dairy shows. Successful Farming,
30 (No. 3):20.

Lush, Jay L. 1935. Observations of European livestock shows. The Cattleman,
21 (No. 11):21--28.

The Agricultural Council. 1935. Denmark agriculture. Copenhagen. 324 pp.

Wentworth, E. N. 1926. Character correlations, livestock judging and selection for
type. Proc. of the Scottish Cattle Breeding Conference for 1925, pp. 195--236.
\end{hangparas}
\index{Shows and fairs|)}