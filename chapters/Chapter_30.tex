\chapter{Registering High Grades}
\label{cha:Lush_Chapter_30}
\index{Breed associations|(}
\index{Breed purity|(}
\index{Registration of grades|(}

For years after the registry associations were first organized, many of
them admitted to registry grade animals with a certain number of topcrosses
of registered sires. Nearly all of the American registry associations
have ceased doing this, and most of those which register a breed
native to other lands never did admit grades to registry in the United
States. Most of the European breeds still register females having three
or four top crosses of registered sires.\footnote{For example, the Shire
Horse Society in Britain admits mares with three topcrosses by registered
Shire stallions. Volume 4 of their ``Grading-Up Register'' in  1944
contained the pedigrees of 49 mares with one topcross and 41 with two
topcrosses, while 760 mares are entered in the contemporary Volume 64 of
the regular Stud Book.} In most cases those breeds also practice selective
registration (see chapter~\ref{cha:Lush_Chapter_16}). The proposal is
occasionally made that American breeds should also admit to registry high
grades which are outstanding individuals.

The pure breeds of today are comparatively modern developments.
Few have herdboks much more than 80 years old. Usually there was
no official herdbook until years after the breed had really been formed.
Naturally, at the time the herdbooks were established, the line between
registered and other animals was somewhat arbitrary. Often there was
disagreement as to whether certain animals really should have been
included in the herdbook. With this background for the history of
registration, the question automatically arises: What is wrong now
with registering grades, when that was done in the founding of all
breeds and still continues among many of them?

\section*{BREED HOMOZYGOSITY}
\index{Homozygosis|(}

The generations of strictly pure breeding that have elapsed since
registration began have enabled the breeders to make the breed somewhat
more homozygous than it was when the herdbooks were first
closed. The amount of \index{Heterozygosis}heterozygosity which was in the original
foundation stock but which has been lost solely through the pure breeding
since pedigree registration began can be measured by studying sample
pedigrees of the breed at any desired date. Those figures for the breeds
so far studied are as follows:

\begin{table}[h]
	\centering
	\begin{tabular}{L{8cm}R{3cm}}
	Shorthorn cattle in Great Britain 				& 26.0\% by 1920 \\
	Jersey cattle in Great Britain 					& 3.9\% by 1925 \\
	Ayrshire cattle in Great Britain 				& 5.3\% by 1927 \\
	Holstein-Friesian cattle in the United States	& 4.0\% by 1931 \\
	Hereford cattle in the United States 			& 8.1\% by 1930 \\
	Brown Swiss cattle in the United States 		& 3.8\% by 1929 \\
	Aberdeen-Angus cattle in the United States 		& 11.3\% by 1939 \\
	Clydesdale horses in Great Britain 				& 6.2\% by 1925 \\
	Rambouillet sheep in the United States 			& 5.5\% by 1926 \\
	Hampshire sheep in the United States 			& 2.9\% by 1935 \\
	Poland-China hogs in the United States			& 9.8\% by 1929 \\
	Brown Swiss cattle in Switzerland				& 1.0\% by 1927 \\
	Landrace swine in Denmark						& 6.9\% by 1930 \\
	Thoroughbred horse in the United States 		& 8.4\% by 1941 \\
	Standardbred horse in the United States			& 4.4\% by 1940 \\
	American Saddle horse							& 3.2\% by 1935 \\
	Telemark cattle in Norway						& 2.3\% in 23 years
	\end{tabular}
\end{table}

\noindent
These figures are probably no more than .5 per cent above or below
what would have been found if it had been possible to study the pedigrees
of all the animals of the breed. Evidently pure breeding by itself
causes only a slow drift toward homozygosity. The high figure for the
Shorthorns was mostly incurred in the first 30 years, most of it while
the breed was largely confined to the herds of the Colling brothers. For
the other breeds it appears that about one-half of 1 per cent ot the
remaining heterozygosity is being lost per animal generation.

The figures given do not include any changes in heterozygosity
which may have been caused by selecting animals which were more or
less homozygous than their pedigrees indicate. Selection changes homozygosi
ty only incidentally as a result of the changes it makes in gene
frequency. Selection requires a long time to change q enough to make
much change in $2q(1 - q)$ except where the genetic situation is simple
and selection is directed toward an extreme and the favored gene
already has a frequency above .5. Selection may have increased materially
the homozygosity of some genes which affect color, distinct
anatomical peculiarities, and other details of breed type for which the
genetic situation may be rather simple. But it is unlikely that selection
has changed very much the average homozygosity of the breeds for
genes affecting complicated characteristics. Selection for the effects of
heterosis may even have operated in the other direction to hold the
heterozygosity at a little higher figure than the pedigree studies indicate.
It seems unlikely that selection can have had much net effect on
the general homozygosity of the breed.
\index{Homozygosis|)}

\nowidow
These considerations make it reasonably certain that the purebreds
are more homozygous than the commercial stock, but doubtful that
their difference in this respect is extreme, even allowing liberally for
the fact that only a restricted group entered the registry in the first
place. Grades having at least four topcrosses\index{Topcrossing} of registered blood would
already be homozygous for about seven-eighths of the genes which were
homozygous in that breed but rare in other animals. The admission of
a few such grades would lower the breed's homozygosity very little.

\section*{INTRODUCING DESIRABLE GENES}

Some grade individuals are much superior to the average of the
purebreds in type or production or both. The best of these grades may
possess desirable genes which are either unknown in the pure breed or
are rare. The admission of these animals to registry might improve the
pure breed through introducing or making more frequent such desirable
genes. If females were required to have at least four topcrosses of
registered sires in order to be registered, only about one-sixteenth of
their genes would be other than those of the breed itself. This fraction
would be further halved in their offspring. The frequency of genes
already existing in the breed would not be changed much through the
admission of such grades. If there are desirable genes which do not now
occur at all within the pure breed, their introduction in this way might
be important.

\section*{GRADES SHOULD MEET HIGH STANDARDS AS INDIVIDUALS}

If the admission of grades is to improve the merit of the breed, the
advantage from introducing desirable genes should be greater than the
damage which would be done by upsetting the extra homozygosity
which the breed has already obtained. Such safeguarding could be
obtained by requiring the grade animal to meet distinctly higher standards
of individuality and production than the average individual merit
of the animals already registered. Just how much higher than the
breed average these standards of individual merit should be for grades
for which registry is being asked would depend in principle upon how
different the pure breed really was from the foundation stock from
which this grade was produced. Any standards adopted would necessarily
be somewhat arbitrary.\footnote{For an example of standards of this kind in
actual operation in Sweden, see \textit{Hoard's Dairyman}, 74:62.}

\section*{DISRUPTING EPISTATIC COMBINATIONS}
\index{Epistatic effects|(}

It is possible that the present merit of the various pure breeds
depends in part upon certain combinations of genes which produce
good results as a combination but not separately. One such combination
might be typical of one breed (although of course not entirely homozygous
in all animals of that breed), while very different combinations
which have the same general kind of effect may be typical of other
breeds. If so, the admission of even a little outside blood to the breed
might scatter those epistatic combinations enough to lower merit more
than the small percentage of outside blood admitted would indicate.

While this is theoretically a possibility, yet there seems no way to
estimate whether such situations are frequent enough to be important.
Even if such situations are frequent, grades carrying as many as four
topcrosses of the pure breed would already have most of the genes
which are necessary for such combinations. That makes the theoretical
danger of harming the breed in this way seem rather remote. The existence
of even a slight possibility of such damage is an additional reason
for requiring that any grades to be admitted should meet standards of
individual merit distinctly higher than the average of the pure breed
into which they come.
\index{Epistatic effects|)}

\section*{ADMISSION OF PUREBREDS WHICH ARE NOT NOW ELIGIBLE TO REGISTRY}

It often happens that a purebred animal cannot be registered
because the breeder is not certain of its sire. The breeder may be certain
that the sire of the animal was one of two or three males, all of
which were purebred, but he may not have any record of the breeding
date, or the actual birth date may be far from the expected one. Economical
use of range resources often requires several males for each
group of females. If the females are all purebred and the males are all
purebred, all the offspring are purebred, too; but their individual
pedigrees are not known and they are not at present eligible to registry
in any association. Sometimes outstanding individuals are produced
from such flocks or herds, and the breeder would like to use them for
stud purposes if they could be registered. Several sheep associations in
the United States have discussed proposals for such \index{Flock registration}
``flock registration,''
but none have been adopted. Such ``flock registration'' would not lower
the homozygosity so far attained in the breed. A breeder using such animals
could still use mass selection in improving his livestock but could
not make much use of ancestors and collateral relatives to estimate
breeding worth. Such a proposal might have unexpected consequences
on the finances of the registry association, but that might be controlled
by adjusting the fees for flock registration. Flock registration is a well
established practice with sheep in many other lands, notably Australia.

\section*{ECONOMIC AND PSYCHOLOGICAL CONSEQUENCES}

To admit high grades to registration might lower the breed reputation
with those who set the highest value on absolute purity of breeding.
This might be offset, at least in part, by the fact that some would
construe such action to mean that individual merit was so highly
respected in this breed that something in pedigree desirability would
be sacrificed to attain unusual individual merit.

The registration of high grades would increase the supply of registered
animals and, therefore, might tend to lower the prices which
could otherwise be obtained by those who already have the purebred
individuals. This could be controlled by having the requirements of
individuality and productive ability so high that only a few grades
would be admitted. Apparently this actually happens abroad. In most
breeds where grades having at least four topcrosses of registered ancestry
may be admitted to registry, only a small number are thus admitted
each year. This slight increase in supply might be more than offset by an
increased demand from those commercial producers who might be
favorably impressed by the evident interest of that breed in individual
merit. Just how those two factors would balance is not clear, but it
seems unlikely that enough high grades would ever be admitted to registry
to be an important economic factor.

Breeders would disagree about the wisdom of admitting any high
grades at all to registration. If any breed does undertake such registration,
it is likely that the official pedigrees will indicate which animals
are absolutely purebred in all lines and which trace to some of those
admitted high grades. Thus, the advocates of absolute purity could
avoid pedigrees tracing in any line to the grades. Something of that kind
actually happens unofficially in the case of pedigrees where there is suspicion
but not absolute proof of fraud. Those who know the situation
let such animals strictly alone, treating them as grades. There is thus a
tendency to eliminate them from the breed, although a beginner unfamiliar
with the situation will sometimes pay purebred prices for them.
After many generation.s an official system of indicating those which
trace at all to grades might break down with its own complexity; but
long before that time had arrived, the breeders would either have
revoked the registration of grades or would have come to substantial
agreement that it was a sound policy.
\index{Breed purity|)}

\section*{SUMMARY}

The registration of high grades unusually superior in individual
merit is a common practice with many European breed associations but
is practiced by very few associations in the United States.

The genetic consequences of such a practice are: (a) Some loss in
homozygosity, and (b) the possibility of introducing some desirable
genes which are rare or unknown in the pure breed.

If the grades were required to meet distinctly higher standards of
individual merit and productivity than the average of the purebreds,
the gain to the breed through the increase of desirable genes would be
likely to be greater than the harm through loss of homozygosity.

Some animals which are absolutely purebred cannot now be registered
because of uncertainty about which of two or more purebred animals
was the sire. The admission of such animals to registry through a
plan of ``herd registration'' or ``flock registration'' would not result in
any loss of the breed's homozygosity. It might lead to a little breed
improvement if only superior individuals were registered, although the
selection of such superior individuals could be little more accurate than
simple mass selection in general.

Economic reasons will predominate in decisions about the registration
of grades. There might be some loss through the breed's appearing
less strict in its standards of purity than its competitors. There might
be some gain through advertising that individual merit was receiving
special attention in that breed.
\index{Breed associations|)}
\index{Registration of grades|)}

\section*{REFERENCES}

For accounts of the increase in homozygosis in various breeds, see
the references at the end of chapter~\ref{cha:Lush_Chapter_21}. For
other articles dealing more directly with the possibilities of
registering high grades or with differences between grades and
purebreeds, see the following:

\begin{hangparas}{0.5in}{1}%
Anonymous. 1927. A challenge to pure breeds. \textit{Hoard's Dairyman},
72:126--27. Feb. 10.

Anonymous. 1929. Admit grades to registry. \textit{Hoard's Dairyman},
74:637. July 10.

McDowell, J.C. 1928. Comparison of purebred and grade dairy cows. USDA,
Cir. 26.

Peterson, Guy A. 1929. Swedish herdbook registration. \textit{Hoard's
Dairyman}, 74:62.

Savage. E. S. 1936. Registration of grade cattle. \textit{Hoard's Dairyman},
8l(9):242. May 10.
\end{hangparas}