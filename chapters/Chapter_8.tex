\chapter{Heredity and Environment}
\label{cha:heredity-and-environment}
\index{Environment and heredity|(}
\index{Heritability|(}

In the strictest sense of the word, the question of whether a characteristic
is hereditary or environmental has no meaning. Every characteristic
is both hereditary and environmental, since it is the end result of a
long chain of interactions of the genes with each other, with the environment
and with the intermediate products at each stage of development.
The genes cannot develop the characteristic unless they have the
proper environment, and no amount of attention to the environment
will cause the characteristic to develop unless the necessary genes are
present. If either the genes or the environment are changed, the characteristic
which results from their interactions may be changed.

Nevertheless, it is often convenient to speak of a characteristic as
``hereditary'' or ``highly hereditary'' when we wish to emphasize that
most of the differences we usually see between individuals in that characteristic
are caused by differences in the genes they have, and only a
few of the differences between individuals are caused by differences in
the environments under which they developed. The difference between
black and red coat color in cattle is such an example of a highly hereditary
characteristic. Environmental circumstances, such as exposure to
sunshine, may cause the black to vary from a jet black to a rusty or
brownish black: but that is a tiny variation compared with the large
difference between black and red which is caused by differences in
genes.

With equal logic it is often convenient to call a characteristic ``environmental''
or ``only slightly hereditary'' when most of the differences
ordinarily found between individuals in that population are caused by
differences in the environments under which they developed and only
a small part of those differences between individuals are caused by differences
in the genes they have. Examples of such largely environmental
characteristics are degrees of fatness and of lameness. In most
populations variations in those are much more apt to have resulted
from previous management, feeding, accidents, or condition of general
health than from differences in heredity. Yet it is certain that some
individuals have genes which make them fatten more readily than others, or
have genes causing structural weaknesses which predispose them to
lameness.

The whole matter of whether a characteristic is hereditary or environmental,
if we find it convenient to state it in that way, is a question
of how much of the variation in that characteristic in that population is
caused by differences in heredity and how much is caused by differences
in environment.

\index{Variance|(}
The question of whether heredity or environment is the more
important can be phrased precisely for a particular trait in a particular
population and, if the data are available, can be answered. It does not
have a single answer true for all traits in one population nor for the
same trait in all populations. Let $\sigma_O^2 =$ the actually observed variance,
$\sigma_H^2 =$ that part of the variance caused by differences in the heredity
which different individuals have, and $\sigma_E^2 =$ that part of the variance
caused by differences in the environments under which different individuals
developed. Then\footnote{In order not to confuse the argument, the nonadditive
interactions of heredity and environment are neglected here. There is reason to
think that those joint effects will generally be small. This definition includes
as ``hereditary'' the dominance and epistatic deviations since they result from
differences between whole genotypes, although they will not contribute so much to
the likeness between relatives as the additive differences do.} $\sigma_H^2 +
\sigma_E^2 = \sigma_O^2$ and $\frac{\sigma_H^2}{\sigma_H^2 + \sigma_E^2} =
\frac{\sigma_H^2}{\sigma_O^2} =$ the portion of the observed variance for which
differences in heredity are responsible. When this fraction is large, we say
that the characteristic is highly hereditary; when this fraction is small, we
call the characteristic slightly hereditary or largely environmental.

The value of $\frac{\sigma_H^2}{\sigma_H^2 + \sigma_E^2}$ can be altered by
changing either $\sigma_H^2$ or $\sigma_E^2$. If we try to make the environment
exactly the same for all individuals, as is usually attempted in genetic experiments,
we may go far in that direction although we can hardly hope to control the environment
perfectly. So far as we succeed in making $\sigma_E^2$ smaller than it is in the
general population, we make the variations in that characteristic more
highly hereditary in our material than they were in the general population.
If we also enlarge $\sigma_H^2$ in our material by mating like to like while
selecting for opposite extremes or by inbreeding in several separate
lines, we increase $\sigma_H^2$ and make variations in that characteristic in our
material still more highly hereditary. Unless we are quite aware of what
we have done in partially controlling the environment (and thus making
$\sigma_E^2$ tend toward zero) and in selecting or breeding to increase the
genetic diversity of our laboratory material, we are apt to get an
exaggerated idea of the importance of heredity in causing variations in that
characteristic in the general population. On the other hand, if we are
experimenting with the effects of some procedure in nutrition or in
management, we will probably try to make $\sigma_H^2$ as small as possible by
using a uniform stock, discarding at the beginning of the experiment
any which appear to deviate much in either direction from the population
average, and minimizing the hereditary differences between lots by
putting litter mates in different lots, etc. Then we will also make our
experimental (environmental) treatments so contrasting that we will be
reasonably sure to find differences in their results. Again, unless we are
quite aware of what we did in neutralizing hereditary differences and in
magnifying environmental differences in our laboratory material, we
are apt to get an exaggerated idea of how important our environmental
differences are in causing the variations in the general population. If
this were clearly understood, a considerable amount of fruitless controversy
would be avoided.

A quantitative statement of the relative importance of heredity and
environment is a partial description of the causes of the variation in a
particular characteristic in the specified population. It is useful in estimating
the probable results of certain breeding systems in the next
generation or two, but it tells nothing about the ultimate limits of the
changes which might be made in that population either by breeding or
by altering its environment.

\section*{METHODS OF ESTIMATING HERITABILITY}
\index{Heritability!methods of estimating|(}

All methods of estimating heritability rest on measuring how much
more closely animals with similar genotypes resemble each other than
less closely related animals do. The techniques suitable for doing this
vary with the material and according to whether environmental correlations
between relatives and the peculiarities of the mating system, if it
was other than random, can be measured and discounted by other
means.

Variation within isogenic lines is wholly environmental. Comparing
this with the variation in an otherwise similar random breeding population
may give an estimate of heritability. This method is of little use in
farm animals because among them are no isogenic lines\index{Isogenic lines} except occasional
pairs of identical twins. These are difficult to identify but if
sought diligently and studied intensively might finally give reliable
information. Identical twins\index{Identical twins} are to be compared with ordinary twins,
rather than with pairs of individuals unrelated to each other, lest the
similarities in the environments of twins might lead to errors in the
interpretation. The method of isogenic lines is the only method likely
to measure all of the epistatic and dominance variations as well as the
additive ones but, because of the rarity of identical twins\index{Identical twins} in farm animals,
is not promising as a source of information fairly free from
sampling errors.

In selection experiments, if we can measure the amount by which
those selected to be the parents exceeded the average of their generation,
we can divide that into the amount by which the average of the
offspring exceeded the average of the generation in which their parents
were born. This gives a measure of the additive portion of the variance
plus a portion (somewhat less than half) of the epistatic variance. In
order not to be misled by unnoticed environmental changes, it is usually
necessary that selection be practiced in opposite directions at the
same time, so that the interpretation will be based on differences
between the high and the low lines, rather than on the absolute values
of the averages. Sometimes this method can be followed in experiments
especially designed for this purpose at some research institution, but it
is not often available to the breeder, since he can rarely afford to select
in the undesired direction just to get information on heritability.

The resemblance between parent and offspring is the most widely
useful method, but is likely to include some environmental correlation
between parent and offspring. Also it will include something from the
resemblance of the offspring to the other parent, if mating were not
random. In using this method the procedure is as follows: (1) observe
the correlation between parent and offspring, (2) subtract from that the
environmental contribution, (3) double the remainder, and (4) divide
by one plus the correlation between mates.\footnote{This should be the
genetic correlation (coefficient of relationship) if the departures
from random mating were of the inbreeding kind, but should be the actually
observed correlation if the mating choices were based on each animal's own
individuality for the characteristic being studied. One will rarely be
certain about this.} The second step is always likely to be difficult,
and the fourth will be unless the deviations from random mating are known
more exactly than is usually the case.

A useful dodge which makes steps two and four unnecessary is to
divide the mates of each sire into a high and a low half on their own
performance, combine the data for all sires, divide the difference
between the daughters of the high and the low halves by the difference
between the two groups of dams and double the results. This measures
the additive portion and a bit of the epistatic portion of the differences
between such dams as were mated to the same sire. It leaves unanalyzed
the differences between the groups of cows which get mated to different
sires. A similar division of the offspring of the high and low sires mated
to the same dam would answer as well in principle but, because of the
usually small number of offspring per dam, is rarely possible with farm
animals.

One of the first clearly analyzed cases of the relative importance of
heredity and environment was Wright's study of the amount of white
spotting in a stock of guinea pigs. It will illustrate the principles and
the use of nearly isogenic lines and of correlation between relatives.
Besides the control stock, in which even second cousin matings had
been avoided, there was a stock which came from the same foundation
but which had been inbred full brother and sister for more than 10
years (probably about 20 to 25 generations in most branches of the family,
nearly all of which came from a single mating in the twelfth generation),
so that it must have been almost entirely homozygous and could
have retained but little genetic variability. By measuring the average
likeness between parents, between parents and offspring, and between
litter mates, Wright was able to separate the variance into a portion
due to heredity, a portion due to environment common to litter mates,
and a remainder due to environment or embryological accidents which
were not alike even for litter mates. Table~\ref{tbl:Lush_Table_6} shows
the findings. The
most illuminating fact for our present purpose is that the variance due
to environment was almost the same in actual units of measurements.
(.354 and .372) in both stocks but was 97.2 per cent of the variance in
inbred stock and only 57.8 per cent of the variance in the control stock.
Here is a case where the same characteristic in two separate stocks
derived from the same colony by different breeding methods is very
slightly hereditary (2.8 per cent) within the inbred stock and nearly
half (42.2 per cent) hereditary within the control stock. All that really
happened was that in one stock the inbreeding had caused nearly all of
the initial hereditary variability to be lost and thereby had altered
greatly the proportion of hereditary to environmental variability.

\begin{table}[htbp]
	\centering
	\caption{\textsc{Piebald Spotting in Guinea Pigs. Portions of the Variance According to Causes
			 (After Wright)}}
	\label{tbl:Lush_Table_6}
	\begin{tabular}{L{3cm}|C{1.5cm}|C{1.5cm}|C{1.5cm}|C{1.5cm}}
		\hline
		\hline
					& \multicolumn{2}{C{3.5cm}|}{$\sigma^2$ in the Inbred Stock}	& \multicolumn{2}{C{3.5cm}}{$\sigma^2$ in the Control Stock} \\
		\cline{2-5}
		Causes of Variance						& Actual Units	& Percentage of Total	& Actual Units	& Percentage of Total \\
 		\hline
		Heredity								& .010			& 2.8					& .271			& 42.4 \\
		Environment common to litter mates		& .020			& 5.5					& .002			& .3 \\
		Environment not common to litter mates	& .334			& 91.7					& .370			& 57.5 \\
		\hline
		Total									& .364			& 100.0					& .643			& 100.0 \\
		\hline
	\end{tabular}
\end{table}

The following are some examples of the kind of analysis which can
be made by comparing correlations between relatives. Gowen studied
Jersey Register of Merit data on milk yield and fat percentage, assuming
that there was no correlation between the environments of daughter
and dam. He came to the conclusion that about 50 to 70 per cent of
the variance in milk production and about 75 to 85 per cent of the
variance in fat percentage came from variations in the heredity of the
individual cows. But if there was as much as .10 to .20 of environmental
correlation betwen [\textit{sic}] daughter and dam, as seems probable from the
usually observed correlation between the records of herd mates, these
figures are too high by .20 to .40. Plum's analysis of the records of cows
in Iowa Cow Testing Associations led him to the figures shown in
Table~\ref{tbl:Lush_Table_7}. Studies of intrasire regression of daughter on dam have
generally given values of around .15 to .30 for heritability of differences in
fat production between cows in the same herd where each cow was
represented by only one record.
\index{Heritability!methods of estimating|)}

\begin{table}[htbp]
	\centering
	\caption{\textsc{Relative Importance of Causes of Variation in Butterfat Production}}
	\label{tbl:Lush_Table_7}
	\begin{tabular}{L{7cm}|L{2cm}L{2cm}}
		\hline
		\hline
		Causes of variation	& \multicolumn{2}{C{4cm}}{Percentage of Total Variance} \\
 		\hline
		Breed											&					& 2 	\\
		Herd											&					&   	\\
		\tabindent Feeding policy of herd				& 12				&		\\
		\tabindent Other causes (genetic or environmental)	& \underline{21}	&		\\
														&					&		\\
														&					& 33	\\
		Cow (mostly genetic)							&					& 26	\\
		Residual (year to year variations)				& 					&		\\
		\tabindent Feeding variations within the herd	& 6					&		\\
		\tabindent Other year to year differences		& 1					&		\\
		\tabindent Length of dry period					& 1					&		\\
		\tabindent Season of calving					& 3					&		\\
		\tabindent Other factors						& \underline{28}	&		\\
														&					& 39	\\
		\hline
		Total											&					& 100	\\
		\hline
	\end{tabular}
\end{table}

Lush, Hetzer, and Culbertson, studying the birth weights of pigs
born during 15 years at the Iowa Agricultural Experiment Station,
came to the figures shown in Table~\ref{tbl:Lush_Table_8}. Part of the 29
per cent due to ``other'' environment common to litter mates may really
have been hereditary as the result of hereditary differences among the dams,
although it was environmental as far as the pigs themselves were concerned.
There is much evidence that the dam's own size or other characteristics
have much influence on the birth weights of her offspring.

\begin{table}[htbp]
	\centering
	\caption{\textsc{Relative Importance of Causes of Variance in Birth Weights of Pigs}}
	\label{tbl:Lush_Table_8}
	\begin{tabular}{L{7cm}|L{2cm}L{2cm}}
		\hline
		\hline
		Causes of variance	& \multicolumn{2}{C{4cm}}{Percentage} \\
 		\hline
		Heredity of the pigs						&					& 	 	\\
		\tabindent Breed differences				& 2					&		\\
		\tabindent Sex								& 1					&		\\
		\tabindent General							& \underline{3}		&		\\
													&					& 6		\\
		Environment common to litter mates			&					&   	\\
		\tabindent Litter size						& 7					&		\\
		\tabindent Year								& 5					&		\\
		\tabindent Ration							& 4					&		\\
		\tabindent Gestation length					& 2					&		\\
		\tabindent Other							& \underline{29}	&		\\
													&					& 47	\\
		Environment not common to litter mates		&					& 47	\\
		\hline
		Total										&					& 100	\\
		\hline
	\end{tabular}
\end{table}

The figures in Tables~\ref{tbl:Lush_Table_7} and \ref{tbl:Lush_Table_8}
illustrate how the actual data may permit subdividing the environmental
variance into portions caused by certain tangible factors or groups of
factors. Nearly always a considerable part of the variance will remain
unidentified as to causes. These are most naturally inferred to have been
individual unobserved (or at least unrecorded) variations in environment,
but in some cases may really have been errors in observations or (in some
methods of analysis) will also have included those portions of the epistatic
or dominance variations which did not contribute to the likeness of the
relatives studied. For other examples similar to Tables~\ref{tbl:Lush_Table_6}
to \ref{tbl:Lush_Table_8}, yet each showing some special features of its own,
see: \textit{Genetics} 19:535; 21:360; 22:468; 26:217; 31:503; and
\textit{Onderstepoort Jour. Vet. Sci. and An. Husb.} 5:580.
\index{Variance|)}

While it is true that the animal at birth contains all the heredity it is
going to have but has not yet been affected by many of the environ mental
circumstances which will affect it, yet the presence or absence of
differences at birth is not a good criterion of whether those differences
are hereditary. Some of the differences found at birth are the result of
previous differences in intra-uterine environment, or of what for lack
of a better term may be called embryological accidents. On the other
hand, many genes in which individuals may differ do not produce their
effects until the individual reaches a certain stage of development.
Examples are the genes which affect early maturity, milk. production,
shape and quality of teeth, and in man such specific things as baldness,
prematurely gray hair, and Huntington's chorea.

\section*{PRACTICAL APPLICATIONS}

Much individual variation is left even when either the heredity or
environment is perfectly controlled. For example, if half of the variance
in a characteristic is hereditary and half is environmental, perfect
control of heredity would still leave the standard deviation 71 per cent
(the square root of one-half) as large as before. If all environmental
variations in a characteristic which is 80 per cent hereditary were eliminated,
the standard deviation would still be nearly 90 per cent (the
square root of 80 per cent) as large as before. If the hereditary variations
were entirely eliminated, the standard deviation would still be about 45
per cent as large as before. Even if a characteristic were 99 per cent
hereditary, complete elimination of all hereditary variability would
still leave the standard deviation 10 per cent as large as it was originally.
Only those variations which are caused by differences in heredity
are themselves inherited. Variations caused by environment can be
large and very important economically, but they do not change the
inheritance of the animal and are not transmitted to its offspring but
must be produced afresh in those offspring by repeating the environmental
treatments which produced them in the parent. There is not
space here to repeat the proofs for the noninheritance of environmental
effects; and, as with other negative concepts, it may be impossible to
prove this one rigorously. But the many experiments carefully planned
to test whether the effects of environmental treatments are inherited in
such a way that the offspring inherit some degree of the modifications
originally produced in their parents by the environmental treatment
have all given negative or doubtful results. Even one who deliberately
wishes to believe that environment does affect heredity in this way must
admit that the effects are so slight that they are not practically important
in any one generation.\footnote{Certain extreme environments, such as exposure
to X-rays, or to barely sub lethal temperatures, or to radium, do increase
the mutation rate} Perhaps they do not occur at all. Both in improving the
heredity and in improving the environment of his animals the breeder is likely
to encounter the law of diminishing returns. Yet in improving their heredity
there is the possibility that if he can achieve enough -- for example, get his
herd widely known as one of the three or four best sources of breeding stock
in the whole breed -- he may come again to a zone of increasing returns because
of the high prices he will receive. The competition to get into that zone is
usually very strong, however.

\index{Acquired characters, noninheritance of|(}Improvements in heredity are
permanent\footnote{Except for gains in the
epistatic effects. Those tend to disappear as the genes
recombine. One must keep on selecting to hold them.} and each generation
stands on the shoulders of the preceding one, whereas improvements in
the environment produce almost their full effect on the animals for
which they are first made. Each new generation must again receive the
improved environment or the gain will be lost. Hence, in the long run
it may be profitable to spend considerable effort to make small improvements
in heredity, since the expense of making such improvements in
one generation may yield dividends for many generations. The expense
of making improvements in heredity (so far as those are additive) is a
capital investment; the expense of making improvements in environment
is an operating expense. Naturally the breeder will wish to do
both so far as they are profitable.\index{Acquired characters, noninheritance of|)}

Besides the economic value of its direct effects on the animals, environment
needs the practical breeder's attention in two ways; First, the
animals should be kept in an environment which will permit them to
show readily which of them come nearest to having all the genes which
have effects the breeder wishes. Second, the breeder should observe
carefully the environment which applies to each animal so that he can
allow for this when making his selections. It is usually impossible to disentangle
the effects of environment completely from the effects of heredity
in individual cases, but some effort spent in trying to do that will
often do much to make selections more accurate and progress more
rapid.

Breeding animals should be kept under environment like that for
which their offspring are being bred. If animals are being bred for
resistance to unfavorable conditions, they should be kept under unfavorable
conditions so that the breeder will have a chance to learn which
ones have the genes that will make them most nearly what he
wants.\footnote{This reaches its extreme form in breeding for disease resistance, a practice to
which is now devoted a large portion of the efforts in plant breeding, but which is
still in the laboratory stage in animal breeding except where (as, for example, in
the tropics or in the breeds of sheep native to marshy regions) some effective natural
selection has been practiced automatically. For fewest mistakes when breeding for
disease resistance, the exposure or inoculation dose should be severe enough that
somewhere between 30 and 70 per cent of the animals would contract the disease.
That may not be practical for diseases which have a high mortality. Those which
cause only a low mortality may not be important enough to warrant much effort in
breeding for disease resistance. Animal breeders now look first of all for prevention,
vaccines, or medicaments as a more economical way out. But if those fail and we are
driven to breed for disease resistance, the above considerations show how far we will
have to depart from the (at present) more orthodox practices of sanitation and
efforts to prevent exposure.} If
cows are being bred for specialized and intensive dairying, they should
be well fed and milked three times a day, provided those are the conditions
under which commercial cows are to be kept in the specialized
dairying of the future. To feed them poorly or milk them only twice a
day would prevent some of the genes, useful under the more specialized
conditions, from showing their presence. On the other hand, to force
the cows by extravagant feeding and by such extreme practices as milking
four times a day would magnify the differences between their production
records and to that extent would be a help in selecting animals
adapted to these conditions; but for most practical purposes this gain
would probably be more than offset by the fact that some of the cows
would respond more than others to those forced conditions, without
there being a corresponding increase in what they would produce under
the usual conditions for which they are being bred. The breeder of
dual-purpose cattle will make fewer mistakes in his selections if his
breeding cows are tested under twice-a-day milking and other management
like that under which he expects their descendants to be used.

The question of testing under forced conditions or under ordinary
conditions can perhaps be clarified by an analogy. If an athletic coach
were to examine all the men in a college to find the best runners for his
track team and were to test them in a race of only one length, such as a
two-mile race, it is true that most of those who did well in this long race
would also do better than average in the short races; yet there would
certainly be some who had not the endurance for the long race but
could do well in the 100-yard dash. There would be others who could
win in the long race but have not the bursts of speed necessary to make
them good performers in the short races. In brief, each man has a certain
amount of general ability to run, and that will manifest itself in
races of all lengths; but each individual also has certain special abilities
or disabilities which will help him or hinder him in certain lengths of
races but not in others. In order to find the very best runners for each
kind of race the coach will need to test them in that kind of race. Yet if
the correlation between their abilities at one length of race and at others
is high, perhaps he might conveniently eliminate half or more of the
whole group from further consideration, by trying them on one kind of
race, without much risk of losing a runner who would be really outstanding
at some of the other lengths of races. Likewise with farm animals
there is doubtless much correlation between an animal's ability to
do well under many different environments, but that correlation is far
from perfect. The genes which enable it to do well or poorly in a certain
environment will manifest themselves most clearly only when the animal
is kept under that environment. In the writings about race horses
there is much mention of ``stayers'' and ``sprinters,'' indicating that
many horses are good in one of these respects but not in the other. Also
they sometimes speak of ``mudders'' which can do well on a wet track
but are outclassed by many others when the track is dry.

If a breeder can foresee that general conditions of management are
going to change in a certain way in the future, it is of course to his
advantage to change his conditions of testing now, so that, when the
general change comes, his animals will already have been selected for a
generation or two toward adaptability to those conditions. In doing so,
of course, he runs the risk that his prediction of the coming change may
be wrong. If it is, his stock may be changed farther away from the real
goal of the future than if he had made no attempt to foresee a change.

Also, a high record made under forced testing has considerable
advertising value. Not all of the potential customers will discount this
high record as much as they should for the environmental circumstances
under which it was made. While this may even be a hindrance
to breed improvement, yet in some cases it has a commercial value to
the individual breeder which he cannot afford to ignore.

Mistaking the effects of the environment for the effects of genes dulls
the keenness of selections, makes the breeder sometimes save animals he
would cull if he knew what genes they had and sometimes cull animals
which have better genes than most of those he saves. This source of
error is very important for such things as growth rate, fertility, health,
vigor in general, size of fleshy parts, ability to fatten, etc., which are
economically important and physiologically complex and often much
modified by environment. The most important practical consequence is
a regression\index{Regression} of the offspring toward the mean of the race. That is, the
offspring of parents which are extreme in either direction will not
usually average as extreme as their parents. The simplest quantitative
expression for this is that for each unit which the selected parents average
above the mean of their race, their offspring will most probably average
about $\frac{\sigma_H^2}{\sigma_H^2 + \sigma_E^2}$ as far above. This would
be literally true if the genes all combined their effects additively. In
actual fact there will be some dominance and some nonadditive gene
interactions which will produce more regression toward the mean of the
race than this formula shows.\footnote{The formula would be more nearly
correct if the numerator were only the additive genetic portion of the
variance; but that is a slight understatement of the case, since a portion
of the epistatic variance also belongs in the numerator.} Mistaking the
effects of environment for the effects of genes, next to matters of health
and fertility in some species, is usually the biggest obstacle to the
breeder's rapid progress toward his chosen goal.

The remedy for confusing the effects of enviroment with those of
heredity is either to control the environment physically by eliminating
variations in it, testing all animals under standard conditions and thus
reducing $\sigma_E^2$ toward zero, or to control the enviroment
statistically by using correction\index{Standardizing records|(} factors to allow as best
one can for unusual individual circumstances when judging what each animal
would have been under standard environmental conditions or what the
difference between two individuals would have been if they had been under
the same environments.
\index{Standardizing records|)}

Physical standardization of the enviroment can never be perfect.
Partial control may be too expensive to carry far. Statistical control
may actually introduce errors through use of the wrong correction factors.
Yet so far as one makes allowances or corrections which are more
often right than they are wrong, he will eliminate more of $\sigma_E^2$ than of
$\sigma_H^2$. Therefore a larger fraction of the variance in his corrected or
adjusted estimates will be hereditary than was true of the variance in
the original observations. S quch allowances for differences in environmental
conditions may vary from the vaguest kind of a mental allowance
to intricate correction factors which sometimes approach the limit
where the increases they make in the accuracy of selections do not pay
for the labor of making the corrections. The man who sees his animals
every day has an important advantage over the man who does not work
with them himself and sees them only at rare intervals or knows them
only through the report of his herdsman, since the former knows the
environmental differences better and can make fairer allowance for
them. The man who works with his animals daily is, however, more
likely to make too much allowance for his favorites without being aware
that he is doing so.

That the offspring of extreme parent~ generally are nearer to the
average of the breed than their parents were (especially as concerns
characteristics of low heritability), does not automatically make the
breed become more uniform as time passes. The offspring from each
parent will vary among themselves and environment will shove some of
them far up and others far down. The extreme ones thus produced will
replenish the supply of extreme individuals in the next generation.

In medical writings a distinction used sometimes to be made
between ``hereditary'' and \index{``Familial''}``familial,'' the former referring to cases
where the offspring was obviously like one parent, and the latter to
traits (such as recessives or those with low ``penetrance''\index{Penetrance}) which ``run in
families'' but in which the individual might not resemble either parent.
Progress in genetic knowledge has now made that distinction obsolete.

``Hereditary'' in the broad sense of the word has nothing to do with
abundance or scarcity of a characteristic or with dominance, although
some methods of estimating heritability do not gather up the variance
due to dominance deviations. Black is no more and no less hereditary
than the much rarer red in black breeds of cattle; rather the question of
heritability concerns the cause of the contrast, black versus red.

\section*{SUMMARY}

\begin{enumerate}
\item All characteristics are both hereditary and environmental in the
strictest sense of those words.
\item Characteristics called hereditary for convenience are those for
which most of the usual differences between individuals are caused by
differences in the genes those individuals have.
\item Characteristics called environmental or nonhereditary are those
for which most of the differences between individuals result from differences
in the environments to which the individuals were exposed.
\item The effects of environment are not inherited except as extreme
environments (like heavy X-ray radiation) produce mutations, and
those are not adaptively related to the environment which produced
them.
\item The breeder should keep his animals under the environments in
which they and their descendants are intended to be used so that the
desired genes may have a chance to express their effects and be recognized
for selection.
\item The breeder will often mistake the effects of environment for
those of genes and will thus make mistakes in his selections. Such mistakes
are usually the most important cause of the fact that the offspring
of selected extreme parents average nearer than their parents to the
mean of their race.
\end{enumerate}
\index{Environment and heredity|)}
\index{Heritability|)}

\section*{REFERENCES}

\begin{hangparas}{0.5in}{1}%
Gowen, John W. 1934. The influence of inheritance and environment on the milk
production and butterfat percentage of Jersey cattle. Jour. Agr. Res. 49:
433--65.

Jennings, H. S. 1930. The biological basis of human nature. 384 pp. New York:
W. W. Norton \& Co. (See especially pp. 127--37, 203--17 and 218--21.)

Lenz, F. 1939. Was bedeutet ``Erblich'' und ``Nicbt-erblich'' beim Menschen? Proc.
Seventh Intern. Cong. of Genet. pp. 187--90. Cambridge.

Lush, Jay L. 1940. Intra-sire correlations or regressions of offspring on dam as a
method of estimating heritability of characteristics. Proc. Amer. Soc. An.
Prod. 293--301.

---; Hetzer, H. O.; and Culbertson, C. C. 1934. Factors affecting birth
weights of swine. Genetics, 19:329--43.

Plum, Mogens. 1935. Causes of differences in butterfat production of cows in Iowa
cow testing associations. Jour. of Dairy Sci., 18:8ll--25.

Wright, Sewall. 1920. The relative importance of heredity and environment in
determining the piebald pattern of guinea pigs. Proc. of the Nat. Acad.
of Sci., 6:320--32.
\end{hangparas}