\chapter[Hermaphroditism and Other Abnormalities]{Hermaphroditism and Other Abnormalities Pertaining to Sex}
\label{cha:Lush_Chapter_35}
\index{Hermaphroditism|(}

The simplest form of reproduction is that in which one organism
merely divides into two and it is impossible to say which is mother and
which is daughter. This is common among protozoa and bacteria. In
the plant kingdom asexual reproduction has been maintained in various
specialized forms (such as budding, sprouting from roots, etc.) even
among the highest plants. None of the higher animals has maintained
asexual reproduction except in the form of parthenogenesis, although
truly remarkable regenerative powers are still possessed by animals as
complicated as many of the worms. Many species of insects, such as the
aphids and the bees, can reproduce parthenogenetically. Sex exists
among them, and sexual reproduction occurs at times, but at other
times the females lay unfertilized eggs which can develop without fertilization
into mature individuals.

\section*{SEXUAL REPRODUCTION}

The union of two individuals to form many others occurs occasionally
among even the simplest protozoa and bacteria. In the simplest
form it is impossible to distinguish which of the uniting individuals is
male and which is female. In such cases the union is usually called by
some such term as ``conjugation.'' In some cases of conjugation it is possible
by physiological means to show that the conjugants are different
in kind and therefore might be said to belong to different sexes. In
some of these cases there are more than two such forms. If they were
called sexes, there would be more than two sexes. Other terms more
precise but less common than male and female are used to describe
such forms.

Sexual reproduction possesses a tremendous evolutionary advantage
over asexual reproduction. In a species reproducing asexually, 100
new mutations would only mean the existence of 101 pure breeding
genotypes from among which natural selection could choose. With sexual
reproduction there is the opportunity for trying out each new mutation
tion in combination with all the others. Therefore, 100 new mutations
in a sexually reproducing species make possible $2^{100}$ new true-breeding
genotypes. This is an enormously greater number. The possibility of
finding some combination among that number which would be superior
to the previous combinations is tremendously increased. This is the
fundamental biological importance of sexual reproduction. It is such a
big advantage that all but the very simplest plants and animals have
evolved ways of reproducing sexually, at least occasionally. Many species,
especially among the plants, have retained asexual reproduction
for part of their life history but at intervals reproduce sexually. This
combines certain advantages of both methods.

\section*{HERMAPHRODITISM}
\index{Sex reversal|(}

Sexual reproduction does not require that the sexes must be in
separate individuals. In many of the simpler animals and in most plants
the same individual has both male and female reproductive organs;
that is, is truly hermaphroditic. Among the vertebrates, only a few fishes
are functionally hermaphroditic, but many animals as highly organized
as the mollusks and round worms are normally hermaphroditic. The
date palm and several of the temperate zone trees, such as the mulberry,
are typical examples of the few higher plants which have the sexes in
separate individuals.\footnote{The botanists call such species ``dioecious''
and restrict the words ``male'' and ``female'' to the gametophyte generation.
This usage docs not correspond to the common and zoological usages of
``male'' and ``female'' but is customary in most botanical writings.} Many
other plants -- hemp for example -- normally exist as separate male and
female individuals; but it is easily possible to reverse the sex or to make
them hermaphroditic by controlling environmental conditions, such as hours
of illumination. Geneticists have even succeeded in producing races of corn
which are dioecious,although corn is typically monoecious, the tassel
bearing the male organs while the ear and its parts are the female organs.

That the animal kingdom has prevailingly adopted sexual reproduction
in a form where the sexes are in separate individuals while the
plant kingdom has prevailingly stayed with hermaphroditism naturally
calls for an explanation. For many species, separate sexes make possible
such a division of labor that a male and female individual together
can leave more descendants than two hermaphroditic individuals could.
Wherever the anatomy and habits of life of the species were such that
division of labor between the sexes conferred this advantage, it was but
natural that the species should ultimately give up hermaphroditism.
Since plants do not move about, they have little to gain by a division of
labor among the two parents. They had much to gain by securing cross-
fertilization, at least occasionally, because permanent and complete
self-fertilization would destroy the genetic advantage of sexual reproduction
in making possible new combinations of genes. The plant world is full
of remarkable mechanisms for promoting cross-fertilization. As long as
plants have the advantage of occasional cross-fertilization, not much
more is to be gained by being dioecious. With the higher animals the
situation is quite different. Many of them take remarkable care of their
young. In most cases much is gained by having one parent specialized
to look after the young directly, the other specialized for
obtaining food, for combat, for protection, and perhaps for other
duties. The animals which have carried this specialization and division
of labor farthest are the social insects, such as ants and bees, with their
workers, soldiers, drones, queens, and other classes. The mode of reproduction
among the mammals is especially extreme in involving considerable
disability of the mother during the bearing and the early rearing
of the young. In most mammals the male has become specialized for
greater efficiency in combat, in the procuring of food, etc., which at
least partly compensates for the fact that his direct contributions to the
rearing of the young are less.
\index{Sex reversal|)}

\section*{SO-CALLED HERMAPHRODITES AMONG MAMMALS}

Cases of partial hermaphroditism\footnote{Colloquially called
``morphadites'' by many stockmen.} are reported frequently in medical
literature. Mammalian embryos all go through early stages in which
it is difficult to be sure of the sex of the individual. That is to say, the
embryos seem to have the anatomical potentiality of developing into
either sex. Which way the development turns is usually determined by
the sex chromosome mechanism, which normally throws the balance
definitely one way or the other. Then, as differentiation proceeds, many
organs develop in a way which is irreversible. Occasionally this balance
is not turned so definitely, and some of the organs develop in one direction
and some in the other. Almost all combinations of this can exist,
but the development of some of the organs precludes the development
of the organs of the other sex so that functional hermaphroditism is
impossible. A few cases of birds which were functional females at one
time in their lives and later became functional males have been
reported. The accessory sex organs in mammals are more specialized,
and the development of those organs is less reversible; therefore, functional
sex-reversal seems impossible among mammals. Even in birds
it is impossible for one individual to be a functional male and a functional
female at the same time.

Probably the things most likely to upset the normal balance of the
sex-determining mechanism are abnormal chromosome divisions or the
action of definite genes on the development of the sex organs. At other
times the cause of the abnormal development is a hormonal disturbance,
such as exists in the case of the free-martin.

Many of the more frequent cases of supposed hermaphroditism are
only incomplete embryological development of some of the sex organs.
Similar embryological accidents sometimes affect parts of the body not
related to sex. Such may be illustrated by the case of hare-lip or cleft
palate, which occurs frequently in man. The upper lip in man develops
embryologically as a center piece and two pieces from the side. These
normally grow together at an early stage. The lines of their union are
still more or less visible in the adult and give the human upper lip its
typical slight approach to a three-lobed condition. Sometimes one or
both of these unions fail to be completed. The result is an individual
with an upper lip divided into two or, in extreme cases, three parts.
This is known as hare-lip. Often the defective union extends back
through the palate and impairs speech. In a similar way, embryological
accidents in the development of the sex organs often result in superficial
appearance of hermaphroditism. The condition called hypospadias,
which occasionally occurs in farm animals, is a case of this. In it the
individual is truly a male but bears some superficial resemblance to a
female and is incapable of reproduction because the urethral groove
does not complete its development to form a closed tube. Most so-called
hermaphrodites among mammals are really males whose development
is imperfect. The extreme cases, of course, are sterile; but some of the
milder cases may be capable of reproduction.

Hermaphroditism which has its origin in definite genes or in chromosomal
disturbances may be inherited to some extent. The actual evidence
of this comes from goats and pigs, where hermaphroditism occurs
more frequently in certain families than in others. Where the hermaphroditism
has its origin in an embryological accident of some kind,
it will not have any hereditary tendency unless the original cause of
the embryological peculiarity was partly hereditary.

\section*{THE FREE-MARTIN}
\index{``Free-martin''|(}

In cattle the developing fetal membranes of twins usually grow
together where they come in contact. If enough of the blood vessels on
the two sets of membranes grow together, the blood of the two twins is
mingled and some of that from each twin actually flows in the blood
vessels of the other. This does no damage when both are males or both
are females; but, when one is a male and the other is a female, the hormones
secreted by the male develop first and exert enough influence on
the female to prevent her normal sex development. The extent to which
her development is altered varies greatly, presumably depending on
how early the blood vessels fuse together and on how complete the
fusion is. Occasionally the membranes do not fuse at all, or least the
blood vessels on them do not grow together, and the male and female
twins born are both quite normal. According to data from 283 such
females tested, only about 1 in 12 among heifers born twins with bulls
are fertile. The bull is not affected, presumably because his sex organs
start to differentiate earlier and never permit those of the female to
reach the stage where they can hamper the male's development. The
free-martin condition is very rare in animals other than cattle. Why the
membranes of twins should so frequently fuse in cattle and so rarely in
other animals is not clear.

The proportion of twin births in cattle is generally low, being something
like 1 twin birth among every 200 births in ``beef cattle'' and 1
twin birth among every 50 to 60 in ``dairy breeds.'' The females of the
unlike-sexed twins will rarely have any other value than that of veal or
beef. The occurrence of a free-martin means practically no loss to the
producer of commercial beef cattle. It means at least a small loss to the
producers of dairy cattle, because normally it costs them more to raise a
cow than the cow will be worth for beef. It is among breeders of purebred
cattle that the free-martin causes the greatest loss. When a heifer
is born twin with a bull the question immediately arises as to whether
it will be profitable to keep her long enough to learn whether she will
be a breeder. The answer will depend in part upon how valuable she
would be if she did prove fertile. If she is of an ordinary family, neither
of her parents nor many of her brothers and sisters being of outstanding
merit, it will probably not be worth while to raise her if that is going to
cost much more than her beef value at maturity. If her value as a
breeding animal would be high, then it might be wise to keep her in the
hope that she will be fertile. For example, suppose it will cost about \$75
to raise this heifer until she is three years old, and that her probable
beef value at that time will be only \$50. The loss incurred in raising
each heifer which proves to be barren would therefore be \$25. If she is
of such valuable breeding and individuality that if fertile she would be
worth \$350 at the age of three years, that would be \$275 more than it
cost to raise her. In the long run one such heifer which did prove fertile
would pay for the loss on 11 which did not. If her breeding value were
still higher, it would be wise to raise her. If her probable mature
breeding value would be only \$125, not enough profit could be made
on the occasional successful case to make up for the many where the
heifer was finally proved barren. That is a rough outline of what should
be considered when one is deciding whether to keep such a twin heifer
to maturity. The situation may be further modified by other evidence.
Many of the free-martins show evidences of abnormality, especially a
distinctly enlarged clitoris or a fold of skin containing a cord along the
median plane of the body just above the rear attachment of the udder,
even at birth. If the individual appears to be abnormal at birth,
attempting to raise it for breeding purposes is almost useless. On the
other hand, if the heifer appears absolutely normal at birth, the chances
of her being fertile are higher than the general average of about 1 in 12.
\index{``Free-martin''|)}

\section*{SCIENTIFIC ASPECTS OF HERMAPHRODITISM}

To the practical breeder all hermaphrodites or partial approaches
to that condition are only annoying sources of loss and are evidence of
nature's blunders. To the physiologist or embryologist they are full of
interest because they may throw light on the interplay of hormones
and organ development and give him new information about this subject.
Nature has performed for him experiments which he could not
perform for himself. For example the free-martin shows him the equivalent
of castration of the female embryo and partial transplantation of
the male gonads at an extraordinarily early stage which he could not
achieve in his laboratory. For this reason the literature of physiology
and embryology has far more cases of this kind reported than corresponds
to the financial importance of the subject to the practical breeder.
\noclub

\section*{SUMMARY}

Sexual reproduction makes it possible for each new mutation to be
tried in combination with all previously existing genes. This is such an
important evolutionary advantage that nearly all species have developed
ways of reproducing sexually, although many of the plants and
some animals as specialized as insects, retain the ability to reproduce
asexually at times. Even in many of the simplest organisms, an occasional
generation of conjugation or sexual reproduction may occur.

The separation of the sexes into different individuals brings advantages
from a division of labor. These are small or non -existent in the
case of most plants but are considerable in nearly all higher animals.

The embryos of the higher animals usually have the anatomical
potentiality of developing into either sex. Normally the sex-chromosomes
throw the balance definitely in one direction or the other. Many
of the changes in the developing sex organs are irreversible. Sometimes
the normal sex-determining balance may be upset by chromosomal
abnormalities or by the effect of definite genes in such a way that development
does not proceed definitely toward the one sex or the other
but toward one sex in some organs and toward the other sex in others.
Various grades of hermaphroditism result. In dioecious plants and in
some of the lower animals this normal balance can sometimes be upset
by certain environmental conditions.

Tendencies to hermaphroditism seem sometimes to be inherited.
The evidence of this among the mammals comes mostly from goats and
pigs.

Many of the cases of supposed hermaphroditism are merely embryological
accidents, usually occurring in an individual which was originally
male but develops imperfectly during its embryology.

The free-martin is an especially interesting case, sometimes regarded
as hermaphroditic but really the result of partial hormonal castration
during embryonic development.
\index{Femininity|)}
\index{Masculinity|)}
\index{Hermaphroditism|)}

\section*{REFERENCE}

\begin{hangparas}{0.5in}{1}%
Allen, Edgar. 1932. Sex and internal secretions. 1,008 pp. Baltimore: William Wood
\& Co.

Crew, F. A. E. 1925. Animal genetics. pp. 187--253. Edinburgh: Oliver and Boyd.

Moore, Carl R. 1941. The influence of hormones on the development of the reproductive
system. Jour. of Urology, 45:869--74.

Swett, W. W.; Matthews, C. A.; and Graves, R. R. 1940. Early recognition of the
free-martin condition in heifers twinborn with bulls. Jour. Agr. Res.
61:587--24.
\end{hangparas}